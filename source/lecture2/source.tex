% (c) Nikita Lisitsa, lisyarus@gmail.com, 2024

\documentclass[10pt,handout]{beamer}

\usepackage[T2A]{fontenc}
\usepackage[russian]{babel}
\usepackage{minted}

\usepackage{graphicx}
\graphicspath{ {./images/} }

\usepackage{adjustbox}

\usepackage{color}
\usepackage{soul}

\usepackage{hyperref}

\usetheme{metropolis}

\definecolor{red}{rgb}{1,0,0}
\definecolor{green}{rgb}{0,0.5,0}
\definecolor{blue}{rgb}{0,0,1}
\definecolor{codebg}{RGB}{29,35,49}
\definecolor{lightbg}{RGB}{253,246,227}
\setminted{fontsize=\footnotesize}

\makeatletter
\newcommand{\slideimage}[1]{
  \begin{figure}
    \begin{adjustbox}{width=\textwidth, totalheight=\textheight-2\baselineskip-2\baselineskip,keepaspectratio}
      \includegraphics{#1}
    \end{adjustbox}
  \end{figure}
}
\makeatother

\title{Фотореалистичный рендеринг\quad\quad\quad\quad\quad\quad \textit{(aka raytracing)}}
\subtitle{Лекция 2: Освещение и тени, отражение и преломление света, металлы и диэлектрики, коррекция цвета}
\date{2024}

\setbeamertemplate{footline}[frame number]

\begin{document}

\frame{\titlepage}

\begin{frame}
\frametitle{Источники света}
\begin{itemize}
\item Свет излучается некоторой \textit{объёмной средой}: металл в лампе накаливания, газ в галогенной лампе, плазма на Солнце
\pause
\item Чтобы честно учитывать такое излучение, нужны очень сложные уравнения \textit{объёмного рендеринга}
\pause
\item Обычно достаточно считать, что свет излучается только поверхностью объекта -- источника света
\pause
\item Всё ещё сложно учитывать, нужны интегральные уравнения (сл. лекция)
\end{itemize}
\end{frame}

\begin{frame}
\frametitle{Источники света}
\begin{itemize}
\item В real-time графике обычно используют аппроксимации:
\pause
\begin{itemize}
\item Бесконечно удалённый (\textit{направленный}) источник света
\pause
\item Бесконечно маленький (\textit{точечный}) источник света
\end{itemize}
\pause
\item В Whitted-style raytracing'е удобно использовать эти же аппроксимации
\end{itemize}
\end{frame}

\begin{frame}[fragile]
\frametitle{Направленный источник света}
\begin{itemize}
\item Моделируют удалённые, мощные источники -- солнце, луна
\pause
\item Задаются вектором направления и интенсивностью (по трём цветовым каналам)
\end{itemize}
\end{frame}

\begin{frame}[fragile]
\frametitle{Два замечания}
\begin{itemize}
\item \textbf{\alert{N.B.:}} Есть два соглашения: использовать вектор \textit{на источник света}, или вектор \textit{направления движения света}. Это противоположные векторы, и в формулах надо учитывать, какой вектор используется. Мы будем использовать вектор \textit{на источник света}.
\pause
\item \textbf{\alert{N.B.:}} Интенсивность источника света, вообще говоря, ничем не ограничена, т.е. может быть сколь угодно большой. Это \textit{количество света}, а не \textit{цвет объекта}, который не может быть больше 1.
\end{itemize}
\end{frame}

\begin{frame}[fragile]
\frametitle{Точечный источник света}
\begin{itemize}
\item Моделируют относительно близкие источники -- костёр, свеча, лампа
\pause
\item Яркость обычно спадает при удалении от источника
\pause
\item Задаются положением в пространстве и функцией зависимости интенсивности от расстояния
\pause
\item Обычно берут такую функцию затухания (\textit{attenuation}) с расстоянием \begin{math}R\end{math}:
\begin{equation*}
\frac{I}{C_0 + C_1 R + C_2 R^2}
\end{equation*}
\pause
\pause
\item \begin{math}I\end{math} -- интенсивность света без учёта затухания (по трём каналам)
\item \begin{math}C_0\neq 0\end{math} предотвращает сингулярность вблизи источника света
\end{itemize}
\end{frame}

\begin{frame}<1-3>[fragile,label=diffuse]
\frametitle{Диффузные поверхности}
\begin{itemize}
\item Как учесть источники света при вычислении цвета пикселя? \pause Зависит от \textit{материала} объекта (подробнее на следующей лекции)
\pause
\item Один из простейших видов материалов -- \textit{диффузная} (она же -- \textit{ламбертова}) поверхность, отражающая падающий свет равномерно во всех направлениях
\pause
\item Если свет приходит из направления \begin{math}{\color{blue}L}\end{math} (это направление \textit{на свет}) на поверхность с цветом \begin{math}\color{green}\operatorname{color}\end{math} и нормалью \begin{math}N\end{math}, то эта поверхность отражает \begin{math}{\color{green}\operatorname{color}}\cdot({\color{blue}L} \cdot N)\end{math} света в сторону камеры
\pause
\item Если \begin{math}{\color{blue}L} \cdot N < 0\end{math}, свет \textit{вообще не попадает} на эту точку (светит с другой стороны поверхности)
\end{itemize}
\end{frame}

\begin{frame}[fragile]
\frametitle{Диффузная поверхность, направленный источник}
\slideimage{diffuse.png}
\end{frame}

\begin{frame}[fragile]
\frametitle{Диффузная поверхность, точечные источники}
\slideimage{point_lights.png}
\end{frame}

\againframe<3->{diffuse}

\begin{frame}[fragile]
\frametitle{Косинус угла падения}
\begin{itemize}
\item Откуда взялся \begin{math}{\color{blue}L} \cdot N\end{math}?
\pause
\item На поверхность площадью \begin{math}A\end{math} падает световой поток с площадью поперечного сечения \begin{math}{\color{blue} \Phi}\end{math} под углом \begin{math}\color{red}\theta\end{math}
\item Тогда \begin{math}{\color{blue} \Phi} = A \cdot {\color{red} \cos \theta}\end{math}
\item \begin{math}\Longrightarrow\end{math} На единицу площади приходится \begin{math}{\color{red}\cos \theta}\end{math} от общей плотности потока
\end{itemize}
\begin{center}
\begin{tikzpicture}
\draw[black,thick] (0.0, 0.0) -- (10.0, 0.0);
\draw[black,thick] (4.5, 0.0) -- (8.5, 4.0);
\draw[black,thick] (5.5, 0.0) -- (9.0, 3.5);
\draw[black,thick] (4.5, -0.25) -- (4.5, 0.25);
\draw[black,thick] (5.5, -0.25) -- (5.5, 0.25);
\draw[blue,thick] (7.5, 3.0) -- (8.0, 2.5);
\draw[red,thick] (5.5, 0.25) arc (90:45:0.25);

\node[black] at (5.0, -0.5) {A};
\node[blue] at (8.0, 3.0) {\begin{math}\Phi\end{math}};
\node[red] at (5.75, 0.5) {\begin{math}\theta\end{math}};

\only<3->{
  \draw[black,thick,-stealth] (5.0, 3.0) -- (5.0, 3.5);
  \draw[blue,thick,-stealth] (5.0, 3.0) -- (5.35, 3.35);

  \node[black] at (5.0, 4.0) {\begin{math}N\end{math}};
  \node[blue] at (5.7, 3.7) {\begin{math}L\end{math}};

  \node[] at (2.0, 2.0) {\begin{math}{\color{red} \cos\theta} = N \cdot {\color{blue} L}\end{math}};
}
\end{tikzpicture}
\end{center}
\end{frame}

\begin{frame}[fragile]
\frametitle{Ambient освещение}
\begin{itemize}
\item Часто в сцене присутствет рассеянный свет `отовсюду': свет от неба, свет от звёзд, свет отразившийся от стен
\pause
\item Правильный способ это учитывать -- решать интегральное уравнение (следующая лекция)
\pause
\item Дешёвый способ это учитывать -- задать интенсивность ambient освещения и применять его ко всем точкам всех объектов (без косинусов, просто умножением на цвет объекта)
\end{itemize}
\end{frame}

\begin{frame}[fragile]
\frametitle{Диффузная поверхность: алгоритм}
\begin{itemize}
\item Строим луч из камеры в текущий пиксель
\pause
\item Находим пересечение со сценой
\pause
\item Вычисляем вклад ambient освещения
\pause
\item Проходимся по всем (точечным и направленным) источникам света
\pause
\begin{itemize}
\item Вычисляем направление на этот источник и его яркость в точке пересечения
\pause
\item Вычисляем освещённость точки этим источником
\end{itemize}
\pause
\item Цвет пикселя -- сумма вкладов всех источников света и ambient освещения
\end{itemize}
\end{frame}

\begin{frame}[fragile]
\frametitle{Диффузная поверхность: много источников света}
\slideimage{many_lights.png}
\end{frame}

\begin{frame}<1-2>[fragile,label=phong]
\frametitle{Модель Фонга}
\begin{itemize}
\item К диффузной модели можно добавить т.н. \textit{specular} компоненту -- эвристическую аппроксимацию света, вышедшего из источника света и отражённого шероховатой поверхностью в направлении камеры
\pause
\item Ambient, diffuse и specular вместе дают \textit{модель Фонга} -- самую популярную простую модель освещения
\pause
\item Мы не будем явно реализовывать specular составляющую, -- вместо этого мы напрямую реализуем эффекты, которые она аппроксимирует
\end{itemize}
\end{frame}

\begin{frame}[fragile]
\frametitle{Модель Фонга}
\slideimage{phong.png}
\end{frame}

\againframe<2->{phong}








\begin{frame}[fragile]
\frametitle{Нормаль к поверхности}
\begin{itemize}
\item Раньше при вычислении пересечения луча с объетом нам было достаточно знать параметр \begin{math}t\end{math}, описывающий точку пересечения
\pause
\item Для рассчёта освещения нам нужна дополнительная информация: нормаль к поверхности в точке пересечения
\pause
\item Вектор нормали \textbf{всегда} должен быть нормирован, многие формулы будут это неявно использовать!
\end{itemize}
\end{frame}

\begin{frame}[fragile]
\frametitle{Нормаль к плоскости}
\begin{itemize}
\item Для плоскости, описываемой уравнением \begin{math}N\cdot V=0\end{math}, вектор \begin{math}N\end{math} и есть нормаль
\end{itemize}
\end{frame}

\begin{frame}[fragile]
\frametitle{Нормаль к параллелепипеду}
\begin{itemize}
\item Зависит от того, на какой грани было ближайшее пересечение:
\pause
\begin{itemize}
\item Если это грань -X (т.е. \begin{math}P_X = -S_X\end{math} в терминах прошлой лекции), то нормаль это вектор \begin{math}(-1, 0, 0)\end{math}
\item Если это грань +X, то нормаль это вектор \begin{math}(1, 0, 0)\end{math}
\item Если это грань -Y, то нормаль это вектор \begin{math}(0, -1, 0)\end{math}
\item И т.д.
\end{itemize}
\end{itemize}
\end{frame}

\begin{frame}[fragile]
\frametitle{Нормаль к параллелепипеду}
\begin{itemize}
\item Чтобы не делать тонну if'ов, можно схитрить: считая, что параллелепипед в центре координат, вычислить \begin{math}\frac{P}{S}\end{math}, где \begin{math}P\end{math} -- точка пересечения, а \begin{math}S\end{math} -- вектор размеров параллелепипеда по трём осям
\pause
\item Максимальная по модулю компонента вектора \begin{math}\frac{P}{S}\end{math} будет равна единице; если занулить две остальные компоненты, то получится искомая нормаль
\end{itemize}
\end{frame}

\begin{frame}[fragile]
\frametitle{Нормаль к сфере}
\begin{itemize}
\item Если сфера имеет радиус 1, то вектор из начала координат (т.е. центра сферы) в точку пересечения \begin{math}P\end{math} и есть нормаль
\pause
\item Если сфера имеет произвольный радиус \begin{math}R\end{math}, то нормаль это вектор \begin{math}P/R\end{math}
\end{itemize}
\end{frame}

\begin{frame}[fragile]
\frametitle{Нормаль к сфере}
\begin{itemize}
\item Сфера описывается уравнением \begin{equation*}f(P) = X^2+Y^2+Z^2-R^2 = 0\end{equation*}
\pause
\item В общем случае, нормаль к поверхности, заданной уравнением \begin{math}f(P)=0\end{math}, это нормированный вектор градиента \begin{equation*}\frac{\nabla f(P)}{|\nabla f(P)|}\end{equation*}
\pause
\item В случае сферы \begin{equation*}\nabla f(P) = (2X, 2Y, 2Z)\end{equation*}
\pause
\item \alert{\textbf{N.B.}} Множитель 2 можно игнорировать -- он всё равно сократится при нормализации вектора
\end{itemize}
\end{frame}

\begin{frame}[fragile]
\frametitle{Нормаль к эллипсоиду}
\begin{itemize}
\item Эллипсоид описывается уравнением \begin{equation*}f(P) = \frac{X^2}{R_X^2}+\frac{Y^2}{R_Y^2}+\frac{Z^2}{R_Z^2}-1 = 0\end{equation*}
\pause
\item Вычислим градиент: \begin{equation*}\nabla f(P) = \left(\frac{2X}{R_X^2}, \frac{2Y}{R_Y^2}, \frac{2Z}{R_Z^2}\right)\end{equation*}
\end{itemize}
\end{frame}

\begin{frame}[fragile]
\frametitle{Другой взгляд на эллипсоид}
\begin{itemize}
\item Эллипсоид с радиусами \begin{math}(R_X, R_Y, R_Z)\end{math} можно рассматривать как единичную сферу (\begin{math}R=1\end{math}), к которой применили неизотропное масштабирование
\begin{equation*}
\begin{pmatrix}
R_X & 0 & 0 \\
0 & R_Y & 0 \\
0 & 0 & R_Z
\end{pmatrix}
\end{equation*}
\pause
\item Точке сферы \begin{math}(X, Y, Z)\end{math} соответствует точка эллипсоида \begin{math}(X\cdot R_X,Y\cdot R_Y,Z\cdot R_Z)\end{math}
\pause
\item Нормаль к сфере в этой точке \begin{math}(X, Y, Z)\end{math} превращается в нормаль к эллипсоиду
\begin{equation*}
\left(\frac{X\cdot R_X}{R_X^2},\frac{Y\cdot R_Y}{R_Y^2},\frac{Z\cdot R_Z}{R_Z^2}\right) = \left(\frac{X}{R_X},\frac{Y}{R_Y},\frac{Z}{R_Z}\right)
\end{equation*}
\end{itemize}
\end{frame}

\begin{frame}[fragile]
\frametitle{Другой взгляд на эллипсоид}
\begin{itemize}
\item Нормаль к сфере в этой точке \begin{math}(X, Y, Z)\end{math} превращается в нормаль к эллипсоиду
\begin{equation*}
\left(\frac{X\cdot R_X}{R_X^2},\frac{Y\cdot R_Y}{R_Y^2},\frac{Z\cdot R_Z}{R_Z^2}\right) = \left(\frac{X}{R_X},\frac{Y}{R_Y},\frac{Z}{R_Z}\right)
\end{equation*}
\pause
\item \begin{math}\Longrightarrow\end{math} При применении неизотропного масштабирования, к нормали нужно применить \textit{обратное} масштабирование
\end{itemize}
\end{frame}

\begin{frame}[fragile]
\frametitle{Нормаль к объекту в общем положении}
\begin{itemize}
\item Если объект был сдвинут, нормаль никак не меняется
\pause
\item Если объект был повёрнут на кватернион \begin{math}Q\end{math}, и мы вычислили нормаль в системе координат объекта (повернув луч на \begin{math}\overline Q\end{math}), её нужно повернуть обратно в мировую систему координат тем же кватернионом \begin{math}Q\end{math}
\end{itemize}
\end{frame}

\begin{frame}[fragile]
\frametitle{Нормаль при произвольном преобразовании}
\begin{itemize}
\item Мы увидели две ситуации:
\pause
\begin{itemize}
\item Если объект повёрнут, к нормали нужно применить \textit{такое же} преобразование
\pause
\item Если объект отмасштабирован, к нормали нужно применить \textit{обратное} преобразование
\end{itemize}
\pause
\item Что происходит?
\end{itemize}
\end{frame}

\begin{frame}[fragile]
\frametitle{Нормаль при произвольном преобразовании}
\begin{itemize}
\item Нормаль \begin{math}N\end{math} в точке поверхности объекта определяется тем, что она перпендикулярна касательной плоскости \begin{math}T\end{math} в этой точке:
\begin{equation*}
\forall V \in T \quad N\cdot V = 0
\end{equation*}
\pause
\item Если к объекту применяется аффинное преобразование (линейное преобразование + сдвиг), к касательным векторам применяется его линейная часть \begin{math}A\end{math}:
\begin{equation*}
V \mapsto AV
\end{equation*}
\pause
\item Преобразованная нормаль \begin{math}N'\end{math} должна быть перпендикулярна преобразованным касательным векторам:
\begin{equation*}
\forall V \in T \quad N'\cdot AV = 0
\end{equation*}
\end{itemize}
\end{frame}

\begin{frame}[fragile]
\frametitle{Нормаль при произвольном преобразовании}
\begin{itemize}
\item По свойствам сопряженного оператора (т.е. транспонированной матрицы)
\begin{equation*}
N'\cdot AV = 0 \quad \Longleftrightarrow \quad A^{T}N' \cdot V = 0
\end{equation*}
\pause
\item Мы знаем, что исходная нормаль перпендикулярна касательным векторам:
\begin{equation*}
A^{T}N' = N \quad\Longrightarrow\quad N' = (A^T)^{-1} N
\end{equation*}
\pause
\item Обращение и транспонирование матрицы коммутируют между собой, поэтому их композицию часто обозначают как
\begin{equation*}
(A^T)^{-1} = (A^{-1})^T = A^{-T}
\end{equation*}
\end{itemize}
\end{frame}

\begin{frame}[fragile]
\frametitle{Нормаль при произвольном преобразовании}
\begin{itemize}
\item Итак, в общем случае, если к объекту применили аффинное преобразование с линейной частью \begin{math}A\end{math}, то нормаль преобразуется посредством матрицы \begin{math}A^{-T}\end{math}
\pause
\item \alert{\textbf{N.B.}}: Нормаль после этого всё равно нужно отнормировать
\pause
\item \alert{\textbf{N.B.}}: Строго говоря, преобразование \begin{math}-A^{-T}\end{math} тоже отвечает нашим требованиям; мы выбираем \begin{math}A^{-T}\end{math} из условия сохранения ориентации пространства
\end{itemize}
\end{frame}

\begin{frame}[fragile]
\frametitle{Нормаль при произвольном преобразовании}
\begin{itemize}
\item Поворот -- ортогональный оператор, т.е. \begin{math}A^T = A^{-1}\end{math} и \begin{math}A^{-T}=A\end{math}
\pause
\item \begin{math}\Longrightarrow\end{math} Преобразование нормалей это такой же поворот
\pause
\begin{equation*}
\end{equation*}
\item Масштабирование -- симметричный оператор (имеет ортогональный базис из собственных векторов), т.е. \begin{math}A^T = A\end{math} и \begin{math}A^{-T}=A^{-1}\end{math}
\pause
\item \begin{math}\Longrightarrow\end{math} Преобразование нормалей это обратное масштабирование
\end{itemize}
\end{frame}

\begin{frame}[fragile]
\frametitle{Нормаль: внутри или снаружи}
\begin{itemize}
\item Описанные формулы для нормалей всегда дают \textit{внешнюю нормаль} к объекту, т.е. смотрящую наружу объекта
\pause
\item Если камера находится внутри объекта, нам нужна \textit{внутренняя нормаль}, смотрящая в противоположную сторону!
\pause
\item Проще всего это учесть по скалярному произведению направления луча \begin{math}D\end{math} и нормали \begin{math}N\end{math}: если \begin{math}D\cdot N < 0\end{math}, то мы смотрим снаружи объекта, иначе -- изнутри
\pause
\item Если мы изнутри объекта, нужно перевернуть нормаль, и сделать пометку, что мы внутри (это будет важно для преломления света)
\pause
\item \alert{\textbf{N.B.}}: С таким подходом получится, что `внутренность' плоскости это всё полупространство с одной стороны от этой плоскости -- так и задумано
\end{itemize}
\end{frame}

\begin{frame}[fragile]
\frametitle{Пересечение луча с геометрией}
\begin{itemize}
\item К этому моменту функция пересечения луча с геометрией должна возвращать как минимум три вещи:
\pause
\begin{itemize}
\item Параметр \begin{math}t\end{math} точки пересечения
\pause
\item Нормаль \begin{math}N\end{math} к поверхности
\pause
\item Пометка, смотрим мы изнутри, или снаружи
\end{itemize}
\pause
\item Удобно завести под это отдельную структуру в коде, и возвращать \mintinline{cpp}|optional<intersection>|
\end{itemize}
\end{frame}







\begin{frame}[fragile]
\frametitle{Тени: теория}
\begin{itemize}
\item Тень -- область сцены, в которую не попадает (заблокирован чем-то) прямой свет из конкретного источника света
\item Свойство точки по отношению к конкретному источнику света
\end{itemize}
\end{frame}

\begin{frame}[fragile]
\frametitle{Тени}
\slideimage{shadows1.png}
\end{frame}

\begin{frame}[fragile]
\frametitle{Тени: теория}
\begin{itemize}
\item Если источник света точечный (или бесконечно удалённый), тень -- бинарное свойство: луч из точки сцены в источник света или пересекает что-то (точка в тени), или нет (точка не в тени) \begin{math}\Longrightarrow\end{math} жёсткие тени (\textit{hard shadows})
\pause
\item Если источник света объёмный, точка сцены может находиться в тени относительно части источника света, и не находиться в тени относительно другой его части \begin{math}\Longrightarrow\end{math} мягкие тени (\textit{soft shadows})
\pause
\begin{itemize}
\item Точки, полностью находящиеся в тени -- \textit{umbra}
\item Точки, частично находящиеся в тени -- \textit{penumbra} (\textit{полутень})
\end{itemize}
\end{itemize}
\end{frame}

\begin{frame}[fragile]
\frametitle{Тени: мягкие vs жёсткие}
\slideimage{shadow-scheme1.png}
\end{frame}

\begin{frame}[fragile]
\frametitle{Мягкие тени}
\slideimage{shadows2.png}
\end{frame}

\begin{frame}[fragile]
\frametitle{Солнечное затмение}
\slideimage{eclipse.png}
\end{frame}

\begin{frame}[fragile]
\frametitle{Тени: теория}
\begin{itemize}
\item Реальные источники света -- объёмные \begin{math}\Longrightarrow\end{math} реальные тени всегда мягкие
\pause
\item Размер и форма penumbra зависит от размеров и формы источника света, а также от расстояния до него (чем дальше, тем меньше penumbra)
\pause
\begin{itemize}
\item В пределе расстояния \begin{math}\rightarrow\infty\end{math} мягкая тень вырождается в жёсткую
\end{itemize}
\end{itemize}
\end{frame}

\begin{frame}[fragile]
\frametitle{Тени: raytracing}
\begin{itemize}
\item Для получения честных мягких теней нужно решать интегральное уравнение (следующая лекция)
\pause
\item Для получения жёсткой тени можно послать дополнительный луч из точки объекта в направлении источника света (\textit{shadow ray}): если он что-то пересёк, то эта точка находится в тени этого источника света
\end{itemize}
\end{frame}

\begin{frame}[fragile]
\frametitle{Жесткие тени: алгоритм}
\begin{itemize}
\item При вычислении цвета пикселя, при учёте вклада конкретного источника света, строим луч из точки объекта в направлении источника света
\pause
\item Пересекаем этот луч со сценой
\pause
\begin{itemize}
\item Если пересечение есть, точка находится в тени, и вклад этого источника света равен нулю
\pause
\item В противном случае вычисляем вклад этого источника света, как и до этого
\end{itemize}
\end{itemize}
\end{frame}

\begin{frame}[fragile]
\frametitle{Жесткие тени}
\slideimage{shadows.png}
\end{frame}

\begin{frame}<1>[fragile,label=shadow_acne]
\frametitle{Жесткие тени: shadow acne}
\begin{itemize}
\item Если в качестве начала shadow ray взять текущую освещаемую точку, возникнет артефакт \textit{shadow acne}
\pause
\item Это связано с тем, что пересечение луча со сценой иногда даёт нам ту же точку, в которой мы вычисляем освещение, а иногда нет -- в сущности, мы видим результат округления floating-point вычислений
\pause
\item В некоторых частных случаях это можно решить, запоминая, какой именно объект мы пересекли (работает только для непрозрачных выпуклых объектов)
\pause
\item Проще всего решить проблему грязным хаком: явно подвинуть начало луча в сторону от поверхности (в направлении луча, или в направлении нормали к поверхности) на маленькое расстояние, скажем, \begin{math}10^{-4}\end{math}
\end{itemize}
\end{frame}

\begin{frame}[fragile]
\frametitle{Shadow acne}
\slideimage{shadow_acne.png}
\end{frame}

\againframe<1->{shadow_acne}

\begin{frame}<1>[fragile,label=point_shadow]
\frametitle{Жесткие тени: точечный источник}
\begin{itemize}
\item При вычислении теней от точечного источника нас интересует только пересечение луча с объектами ближе источника света, иначе получатся неправильные тени
\pause
\item Если два способа это решить:
\pause
\begin{itemize}
\item 1. После вычисления пересечения shadow ray со сценой сравнить параметр \begin{math}t\end{math}пересечения с расстоянием до источника света: если источник ближе, считаем, что точка \textbf{не в тени}
\pause
\item 2. Передавать в функцию вычисления пересечения луча со сценой максимальное расстояние вдоль луча как дополнительный параметр (в обычной ситуации равный \begin{math}+\infty\end{math}): все пересечения с \begin{math}t > max\end{math} нужно игнорировать
\end{itemize}
\pause
\item Лучше сделать второй вариант: в дальнейшем он будет нам полезен для оптимизаций
\end{itemize}
\end{frame}

\begin{frame}[fragile]
\frametitle{Неправильные тени от точечного источника}
\slideimage{wrong_shadow.png}
\end{frame}

\againframe<1->{point_shadow}





\begin{frame}[fragile]
\frametitle{Более сложные материалы}
\begin{itemize}
\item Диффузная модель хорошо описывает матовые, шероховатые поверхности
\pause
\item Далеко не все поверхности в реальном мире -- диффузные!
\pause
\item В графике обычно рассматривают два основных вида нетривиальных поверхностей: \textit{металлы} и \textit{диэлектрики}
\pause
\item Более сложные материалы часто строятся как комбинация диффузной, металлической, и диэлектрической моделей
\end{itemize}
\end{frame}

\begin{frame}[fragile]
\frametitle{Металлы}
\begin{itemize}
\item Особенность металлов в том, что они \textit{отражают} приходящий свет
\pause
\item То, сколько света отразится, определяется \textit{цветом} металла
\pause
\item Отражения от шероховатого металла -- размытые (в пределе получится диффузная поверхность), и чтобы честно их учитывать, нужно, -- вы не поверите! -- решать интегральное уравнение
\pause
\item Идеальные (неразмытые) отражения легко получить, послав дополнительный луч
\end{itemize}
\end{frame}

\begin{frame}[fragile]
\frametitle{Направление отражённого луча}
\begin{itemize}
\item Пусть свет пришёл из направления \begin{math}L\end{math} в точку с нормалью \begin{math}N\end{math}
\pause
\item Направление отражённого луча \begin{math}R\end{math} можно вычислить по формуле 
\begin{equation*}
R = 2N\cdot(N \cdot L) - L
\end{equation*}
\pause
\item У нас \begin{math}L=-D\end{math}, где \begin{math}D\end{math} -- направление луча из камеры, и формула переписывается как
\begin{equation*}
R = D - 2N\cdot(N \cdot D)
\end{equation*}
\pause
\item \alert{\textbf{N.B.}}: Как и с теневым лучом, начало отражённого луча нужно немного подвинуть в сторону \textit{от} поверхности
\end{itemize}
\end{frame}

\begin{frame}[fragile]
\frametitle{Металлическая поверхность: алгоритм}
\begin{itemize}
\item Строим луч из камеры в текущий пиксель
\pause
\item Находим пересечение со сценой
\pause
\item Строим отражённый луч
\pause
\item Рекурсивно вызываем функцию получения света по лучу в направлении отражённого луча
\pause
\item Цвет пикселя -- полученный рекурсивным вызовом свет, умноженный на цвет металлического объекта
\end{itemize}
\end{frame}

\begin{frame}[fragile]
\frametitle{Металлическая поверхность}
\slideimage{reflection.png}
\end{frame}

\begin{frame}[fragile]
\frametitle{Металлическая поверхность: рекурсия}
\begin{itemize}
\item Мы впервые столкнулись с необходимостью \textit{рекурсивно} вызвать функцию получения света по лучу
\pause
\item В общем случае ничто не гарантирует, что эта рекурсия закончится за разумное число шагов
\pause
\item Стандартное решение -- ограничить максимальную глубину рекурсии некоторым числом, обычно от 4 до 16
\end{itemize}
\end{frame}





\begin{frame}<1-2>[fragile,label=dielectric]
\frametitle{Диэлектрики}
\begin{itemize}
\item Диэлектрик -- вещество, не проводящее ток (в физике определение не совсем такое, но нам это не важно)
\pause
\item Особенность диэлектриков в том, что они как отражают, так и преломляют свет
\pause
\item За эти процессы отвечают закон Снеллиуса и уравнения Френеля
\pause
\item \alert{\textbf{N.B.}}: Непрозрачные объекты -- часто тоже диэлектрики, но эффекты поглощения и рассеяния света внутри этих объектов делают их непрозрачными; обычно такие объекты моделируют как диффузные с отражающим покрытием, описывающимся уравнениями Френеля
\end{itemize}
\end{frame}

\begin{frame}[fragile]
\frametitle{Диэлектрики}
\slideimage{dielectric.jpg}
\end{frame}

\againframe<2->{dielectric}

\begin{frame}<1-2>[fragile,label=dielectric]
\frametitle{Закон Снеллиуса (Snell's Law)}
\begin{itemize}
\item Пусть свет идёт из среды с коэффициентом преломления \begin{math}\eta_1\end{math} в среду с коэффициентом \begin{math}\eta_2\end{math}
\pause
\item Тогда угол падения \begin{math}\theta_1\end{math} (между падающим лучом и нормалью к поверхности) и угол преломления \begin{math}\theta_2\end{math} связаны соотношением

\begin{equation*}
\eta_1 \cdot \sin \theta_1 = \eta_2 \cdot \sin \theta_2
\end{equation*}
\end{itemize}
\end{frame}

\begin{frame}[fragile]
\frametitle{Закон Снеллиуса (Snell's Law)}
\slideimage{snells_law.png}
\end{frame}

\begin{frame}
\frametitle{Коэффициент преломления}
\begin{itemize}
\item Коэффициент преломления (\textit{refractive index} или \textit{index of reflection} -- \textit{IOR}) связан со скоростью распространения света в среде
\pause
\item Типичные значения:
\begin{itemize}
\item Воздух: 1
\item Вода: 1.333
\item Стекло: 1.5
\item Алмаз: 2.4
\end{itemize}
\end{itemize}
\end{frame}

\begin{frame}[fragile]
\frametitle{IOR = 1}
\slideimage{ior_100.png}
\end{frame}

\begin{frame}[fragile]
\frametitle{IOR = 1.01}
\slideimage{ior_101.png}
\end{frame}

\begin{frame}[fragile]
\frametitle{IOR = 1.02}
\slideimage{ior_102.png}
\end{frame}

\begin{frame}[fragile]
\frametitle{IOR = 1.05}
\slideimage{ior_105.png}
\end{frame}

\begin{frame}[fragile]
\frametitle{IOR = 1.1}
\slideimage{ior_110.png}
\end{frame}

\begin{frame}[fragile]
\frametitle{IOR = 1.3}
\slideimage{ior_130.png}
\end{frame}

\begin{frame}[fragile]
\frametitle{IOR = 1.5}
\slideimage{ior_150.png}
\end{frame}

\begin{frame}[fragile]
\frametitle{IOR = 2}
\slideimage{ior_200.png}
\end{frame}

\begin{frame}
\frametitle{Направление преломлённого луча}
\begin{itemize}
\item Из закона Снеллиуса можно вывести формулу для направления преломлённого луча
\pause
\item Пусть свет падает из направления \begin{math}L\end{math} на поверхность с нормалью \begin{math}N\end{math}
\pause
\item Тогда направление преломлённого луча \begin{math}R\end{math} можно вычислить как
\begin{equation*}
\begin{gathered}
\cos\theta_1 = N\cdot L \\
\sin\theta_2 = \frac{\eta_1}{\eta_2}\sqrt{1-\cos^2\theta_1} = \frac{\eta_1}{\eta_2}\sqrt{1-(N\cdot L)^2} \\
\cos\theta_2 = \sqrt{1 - \sin^2\theta_2} \\
R = \frac{\eta_1}{\eta_2}(-L) + \left(\frac{\eta_1}{\eta_2}\cos\theta_1 - \cos\theta_2\right)\cdot N
\end{gathered}
\end{equation*}
\end{itemize}
\end{frame}

\begin{frame}
\frametitle{Направление преломлённого луча}
\begin{itemize}
\item Нас интересует, откуда придёт свет, который после преломления попадёт в направлении камеры
\pause
\item \begin{math}\Longrightarrow\end{math} Для луча из камеры в направлении \begin{math}D\end{math} в формулах прошлого слайда нужно взять \begin{math}L = -D\end{math}
\pause
\item \begin{math}\eta_1\end{math} положим равным 1 (воздух)
\pause
\item \begin{math}\eta_2\end{math} -- часть описания материала
\pause
\item Если луч пришёл изнутри объекта, \begin{math}\eta_1\end{math} и \begin{math}\eta_2\end{math} нужно \textbf{поменять местами}!
\end{itemize}
\end{frame}

\begin{frame}
\frametitle{Полное внутреннее отражение}
\begin{itemize}
\item Посмотрим внимательнее на формулу
\begin{equation*}
\sin\theta_2 = \frac{\eta_1}{\eta_2}\sqrt{1-\cos^2\theta_2} = \frac{\eta_1}{\eta_2}\sqrt{1-(N\cdot L)^2}
\end{equation*}
\pause
\item Если \begin{math}\eta_1 > \eta_2\end{math}, значение \begin{math}\sin\theta_2\end{math} может оказаться больше 1 по модулю
\pause
\item В этом случае преломления нет, и объект только отражает свет -- происходит \textit{полное внутреннее отражение}
\end{itemize}
\end{frame}

\begin{frame}[fragile]
\frametitle{Полное внутреннее отражение}
\slideimage{turtle.jpg}
\end{frame}

\begin{frame}
\frametitle{Уравнения Френеля}
\begin{itemize}
\item Описывают то, сколько света отразится, а сколько -- пройдёт сквозь объект (и преломился по закону Снеллиуса)
\pause
\item Сложные уравнения, учитывающие поляризацию света
\pause
\item В графике почти всегда считается, что свет не поляризован
\pause
\item Вместо настоящих уравнений Френеля используется \textit{аппроксимация Шлика}
\end{itemize}
\end{frame}

\begin{frame}
\frametitle{Аппроксимация Шлика (Schlick's approximation)}
\begin{itemize}
\item Пусть свет падает из направления \begin{math}L\end{math} на диэлектрическую поверхность с нормалью \begin{math}N\end{math}
\pause
\item Тогда отразится свет в количестве
\begin{equation*}
R = R_0 + (1 - R_0) \cdot (1 - (N \cdot L))^5
\end{equation*}
\pause
\item Соответственно, преломится свет в количестве \begin{math}1-R\end{math}
\pause
\item Здесь, \begin{math}R_0\end{math} -- коэффициент отражения, если смотреть прямо на поверхность; он равен
\begin{equation*}
R_0 = \left(\frac{\eta_1-\eta_2}{\eta_1+\eta_2}\right)^2
\end{equation*}
\end{itemize}
\end{frame}

\begin{frame}
\frametitle{Диэлектрики: алгоритм}
\begin{itemize}
\item Строим луч из камеры в текущий пиксель
\pause
\item Находим пересечение со сценой
\pause
\item Вычисляем \begin{math}\sin\theta_2\end{math}
\pause
\item Если он \begin{math}\leq 1\end{math}:
\pause
\begin{itemize}
\item Вычисляем преломлённый и отражённый лучи
\pause
\item Рекурсивно получаем свет, приходящий из этих направлений
\pause
\item Цвет пикселя -- среднее этих значений с коэффициентами согласно аппроксимации Шлика
\end{itemize}
\pause
\item Если он \begin{math}> 1\end{math}:
\pause
\begin{itemize}
\item Полное внутреннее отражение, цвет пикселя это отражённый свет
\end{itemize}
\end{itemize}
\end{frame}

\begin{frame}
\frametitle{Диэлектрики: цвет}
\begin{itemize}
\item Цвет полупрозрачного объекта определяется тем, как он окрашивает \textit{проходящий сквозь себя} свет
\pause
\item Правильно было бы учитывать то, как далеко свет шёл внутри полупрозрачного объекта, пока не вышел из него, и использовать формулы из volume rendering'а
\pause
\item Пока будем использовать аппроксимацию: если мы смотрим на диэлектрик снаружи, то преломлённый свет (но \textbf{не отражённый}) нужно умножить на цвет объекта
\end{itemize}
\end{frame}





\begin{frame}<1>[fragile,label=gamma]
\frametitle{Гамма}
\begin{itemize}
\item Пока мы не задумывались о том, что за значения мы записываем в пиксели выходной картинки
\pause
\item Обычно, интенсивность света \begin{math}I\end{math}, излучаемого монитором, нелинейно зависит от значения \begin{math}V\end{math}, записанного в пикселе
\pause
\item Это лучше соответствует восприятию света человеком и даёт больше точности тёмным цветам, которые человек различает лучше
\pause
\item Почти всегда используется показательная функция:
\begin{equation}I \sim V^\gamma\end{equation}
\pause
\item \begin{math}\gamma\end{math} обычно равна 2.2 (некоторые компьютеры Macintosh использовали 1.8)
\end{itemize}
\end{frame}

\begin{frame}
\frametitle{Линейное значение пикселя vs линейная интенсивность излучения}
\begin{figure}
\slideimage{gamma-scale.png}
\end{figure}
\end{frame}

\againframe<2->{gamma}

\begin{frame}[fragile,label=gamma2]
\frametitle{Проблемы гаммы}
\begin{itemize}
\item Искажается восприятие относительных яркостей, особенно при реалистичном рендеринге (e.g. объект в два раза ярче не будет выглядеть в два раза ярче)
\pause
\item Неправильно выглядит освещение, наложение источников света, и т.д.
\end{itemize}
\end{frame}

\begin{frame}[fragile]
\frametitle{Коррекция гаммы (gamma-correction)}
\begin{itemize}
\item Коррекция гаммы -- общий термин для применения любых нелинейных преобразований над интенсивностью пикселя
\pause
\item В рендеринге под гамма-коррекцией обычно подразумевают применение обратного к \begin{math}V^{2.2}\end{math} преобразования, чтобы получить линейную зависимость выходящего излучения от значения пикселя
\pause
\item В нашем случае сводится к тому, что перед записью вычисленного цвета в пиксель, его нужно возвести в степень: \begin{math}V^{1/2.2}\end{math} (\textbf{до} конвертации в диапазон 0..255)
\end{itemize}
\end{frame}

\begin{frame}[fragile]
\frametitle{Эффекты коррекции гаммы}
\slideimage{gamma-ex1.png}
\end{frame}

\begin{frame}[fragile]
\frametitle{Эффекты коррекции гаммы}
\slideimage{gamma-ex2.png}
\end{frame}

\begin{frame}[fragile]
\frametitle{Эффекты коррекции гаммы}
\slideimage{gamma-ex3.png}
\end{frame}

\begin{frame}[fragile]
\frametitle{Эффекты коррекции гаммы}
\begin{itemize}
\item Типичные эффекты гамма-коррекции:
\pause
\begin{itemize}
\item Картинка становится ярче (чем темнее цвет, тем больше прирост яркости)
\pause
\item Картинка становится менее контрастной
\pause
\item Освещение выглядит правильнее
\pause
\item Переход от освещённой к неосвещённой области объекта выглядит более резким (как в реальности)
\end{itemize}
\pause
\item Некоторые tone-mapping кривые \textbf{уже включают} в себя гамма-коррекцию!
\end{itemize}
\end{frame}







\begin{frame}[fragile]
\frametitle{HDR}
\begin{itemize}
\item Проблема: после вычисления освещённости пикселя мы можем получить значения компонент цвета большие, чем 1, и при записи в пиксель они будут обрезаны до 1
\pause
\item В реальном мире интенсивность света ничем не ограничена, и между солнечным днём и тёмной ночью может меняться на 15 порядков
\pause
\item Объект может освещаться тусклым ambient освещением, а может сразу десятком ярких источников света
\end{itemize}
\end{frame}

\begin{frame}[fragile]
\frametitle{HDR}
\begin{itemize}
\item High Dynamic Range (HDR) -- термин, означающий большой (не ограниченный \begin{math}[0, 1]\end{math}) диапазон интенсивностей света
\pause
\item HDR текстура (например, environment map) -- содержит значения, выходящие за диапазон \begin{math}[0, 1]\end{math} (например, 32-bit floating-point)
\pause
\item HDR рендеринг -- позволяет отобразить HDR-освещение
\end{itemize}
\end{frame}

\begin{frame}[fragile]
\frametitle{Tone mapping}
\begin{itemize}
\item Чтобы поддержать HDR рендеринг, нужно превратить диапазон освещённостей \begin{math}[0, \infty)\end{math} в диапазон \begin{math}[0, 1]\end{math}
\pause
\item В принципе, подойдёт любая монотонно возрастающая функция \begin{math}[0, \infty)\rightarrow [0, 1]\end{math} -- \textit{tone-mapping curve}:
\begin{itemize}
\item \begin{math}x \mapsto \frac{x}{1+x}\end{math} -- \textit{Reinhard operator}
\item \begin{math}x \mapsto \arctan(x)\end{math}
\item \begin{math}x \mapsto \frac{2}{1+e^{-x}}-1\end{math}
\end{itemize}
\pause
\item Иногда используют более сложные функции, взятые из кинематографа (filmic tone mapping)
\pause
\item Часто всё равно ограничивают диапазон входных интенсивностей каким-то настраиваемым значением \begin{math}W\end{math} (\textit{white}), и делают нелинейное преобразование \begin{math}[0, W] \rightarrow [0, 1]\end{math}
\end{itemize}
\end{frame}

\begin{frame}[fragile]
\frametitle{Reinhard operator}
\slideimage{reinhard.png}
\end{frame}

\begin{frame}[fragile]
\frametitle{Filmic tone mapping curve (ACES)}
\slideimage{aces.png}
\end{frame}

\begin{frame}[fragile]
\frametitle{Рациональная аппроксимация ACES}
\usemintedstyle{solarized-light}
\begin{minted}[bgcolor=lightbg]{cpp}
vec3 saturate(vec3 const & color)
{
  return clamp(color, vec3(0.0), vec3(1.0));
}

glm::vec3 aces_tonemap(glm::vec3 const & x)
{
  const float a = 2.51f;
  const float b = 0.03f;
  const float c = 2.43f;
  const float d = 0.59f;
  const float e = 0.14f;
  return saturate((x*(a*x+b))/(x*(c*x+d)+e));
}
\end{minted}
\pause
\begin{itemize}
\item Мы будем использовать именно её
\end{itemize}
\end{frame}

\end{document}
