% (c) Nikita Lisitsa, lisyarus@gmail.com, 2024

\documentclass[10pt]{beamer}

\usepackage[T2A]{fontenc}
\usepackage[russian]{babel}
\usepackage{minted}

\usepackage{graphicx}
\graphicspath{ {./images/} }

\usepackage{adjustbox}

\usepackage{color}
\usepackage{soul}

\usepackage{hyperref}

\usetheme{metropolis}

\definecolor{red}{rgb}{1,0,0}
\definecolor{green}{rgb}{0,0.5,0}
\definecolor{blue}{rgb}{0,0,1}
\definecolor{codebg}{RGB}{29,35,49}
\definecolor{lightbg}{RGB}{253,246,227}
\setminted{fontsize=\footnotesize}

\makeatletter
\newcommand{\slideimage}[1]{
  \begin{figure}
    \begin{adjustbox}{width=\textwidth, totalheight=\textheight-2\baselineskip-2\baselineskip,keepaspectratio}
      \includegraphics{#1}
    \end{adjustbox}
  \end{figure}
}
\makeatother

\title{Фотореалистичный рендеринг\quad\quad\quad\quad\quad\quad \textit{(aka raytracing)}}
\subtitle{Лекция 1: Введение в курс. Метод трассировки лучей. Пересечение луча с геометрическими примитивами.}
\date{2024}

\setbeamertemplate{footline}[frame number]

\begin{document}

\frame{\titlepage}

\begin{frame}
\frametitle{}
\begin{itemize}
\item Лисица Никита Игоревич (Яндекс)
\item \texttt{lisyarus@gmail.com}
\item \texttt{+7 (952) 276-70-50}
\end{itemize}
\end{frame}

\begin{frame}
\frametitle{Как устроен курс}
\begin{itemize}
\item \textbf{Лекции}
\begin{itemize}
\item Слайды в репозитории \href{https://github.com/lisyarus/raytracing-course-slides/tree/trunk/pdf}{\texttt{github.com/lisyarus/raytracing-course-slides/tree/trunk/pdf}}
\end{itemize}
\pause
\item \textbf{Практики}
\pause
\begin{itemize}
\item Слайды с заданиями в репозитории \href{https://github.com/lisyarus/raytracing-course-slides/tree/trunk/pdf}{\texttt{github.com/lisyarus/raytracing-course-slides/tree/trunk/pdf}}
\pause
\item Каждая практика -- одно большое задание
\pause
\item Сдача автотестированием + код-ревью
\pause
\item Каждая практика основывается на коде \textit{предыдущей практики}!
\end{itemize}
\end{itemize}
\end{frame}

\begin{frame}
\frametitle{Баллы}
\begin{itemize}
\pause
\item \textbf{Все баллы} получаются за практические задания
\pause
\begin{itemize}
\item \textbf{до 9 баллов} -- 1ая практика
\item \textbf{до 10 баллов} -- 2ая практика
\item \begin{math}\dots\end{math}
\item \textbf{до 16 баллов} -- 8ая практика
\end{itemize}
\pause
\item В сумме -- \textbf{100 баллов}
\end{itemize}
\end{frame}

\begin{frame}
\frametitle{Баллы}
\begin{itemize}
\item Дедлайн для практик -- 1 неделя, т.е. 23:59 следующего понедельника
\pause
\begin{itemize}
\item На 1ую практику дедлайн -- 3 недели
\item На 2ую практику дедлайн -- 2 недели
\end{itemize}
\pause
\item При сдаче после дедлайна баллы делятся \textbf{пополам} (с округлением в большую сторону)
\end{itemize}
\end{frame}

\begin{frame}
\frametitle{Оценка за курс}
\pause
\begin{itemize}
\item Зачет: \textbf{50 и более} баллов
\pause
\item Экзамен:
\begin{itemize}
\item \textbf{50-59} баллов: \textbf{E}
\item \textbf{60-69} баллов: \textbf{D}
\item \textbf{70-79} баллов: \textbf{C}
\item \textbf{80-89} баллов: \textbf{B}
\item \textbf{90-100} баллов: \textbf{A}
\end{itemize}
\end{itemize}
\end{frame}

\begin{frame}
\frametitle{Автотестирование}
\begin{itemize}
\item Каждая практика -- программа, по описанию входной сцены генерирующая картинку (\textit{`рендер'})
\pause
\item Автотестированием занимается телеграм-бот \texttt{mkn-raytracing-2024-bot}
\pause
\item Он запускает вашу программу на 5 \textit{случайных} сценах и сравнивает картинку с референсом
\pause
\item Баллы вычисляются на основе \textit{худшего} процента совпадения вашей картинки и картинки-референса
\end{itemize}
\end{frame}

\begin{frame}[fragile]
\frametitle{Автотестирование}
\begin{itemize}
\item Можно использовать \textit{любой язык программирования} (если он не поддерживается -- напишите мне, я постараюсь его добавить)
\pause
\item \alert{\textbf{N.B.}}: Наши программы будут довольно медленными, так что лучше выбрать язык пошустрее
\pause
\item Нам нужно будет
\begin{itemize}
\item Читать и писать в текстовые и бинарные файлы
\pause
\item Работать с векторами и матрицами, -- можно взять готовую библиотеку (напр. \texttt{glm}), а можно написать все типы и операции самим (нам нужно совсем немного)
\pause
\item Генерировать случайные числа
\pause
\item Оперировать древовидными структурами данных
\end{itemize}
\end{itemize}
\end{frame}

\begin{frame}[fragile]
\frametitle{Автотестирование}
\begin{itemize}
\item Боту нужно послать команду вида
\usemintedstyle{solarized-light}
\begin{minted}[bgcolor=lightbg]{bash}
/submit practice1 https://github.com/vanya/practice1.git
  [ path-inside-repo [ branch ] ]
\end{minted}
\pause
\item Можно делать все задания в одном репозитории или в разных, как хотите
\pause
\item В репозитории по указанному пути должны быть скрипты \texttt{build.sh} и \texttt{run.sh}:
\begin{itemize}
\item \texttt{build.sh} запускается один раз для сборки проекта
\item \texttt{run.sh} запускается для каждого теста, он должен принимать пути до входного и выходного файла
\end{itemize}
\end{itemize}
\end{frame}

\begin{frame}[fragile]
\frametitle{Автотестирование}
Пример для проекта на C++ и CMake:
\usemintedstyle{solarized-light}
\begin{itemize}
\item \texttt{build.sh}
\begin{minted}[bgcolor=lightbg]{bash}
#!/usr/bin/env bash

mkdir build
cd build
cmake .. -DCMAKE_BUILD_TYPE=Release
cmake --build .
\end{minted}
\item \texttt{run.sh}
\begin{minted}[bgcolor=lightbg]{bash}
#!/usr/bin/env bash

./build/solution "$1" "$2"
\end{minted}
\end{itemize}
\end{frame}

\begin{frame}
\frametitle{Пререквезиты}
\pause
\begin{itemize}
\item \textbf{Программирование}
\pause
\begin{itemize}
\item Основы программирования -- циклы, массивы, функции, структуры, и т.п.
\pause
\item Какой-нибудь язык программирования
\pause
\item Компилировать и запускать программы в удобной вам среде
\end{itemize}
\pause
\item \textbf{Математика}
\pause
\begin{itemize}
\item Линейная алгебра (векторы, матрицы, умножение матриц, линейные системы, ортогональность)
\pause
\item Аналитическая геометрия (координаты, уравнения кривых и поверхностей)
\pause
\item Анализ (производные, интегралы, трансцендентные функции)
\pause
\item Основы статистики
\end{itemize}
\end{itemize}
\end{frame}

\begin{frame}<1-2>[label=stages]
\frametitle{Примерный план курса}
\pause
\begin{itemize}
\item \textbf{Лекции 1-3: Whitted-style raytracing}
\begin{itemize}
\item Трассировка лучей без решения уравнения рендеринга
\end{itemize}
\pause
\item \textbf{Лекции 4-6: Monte-Carlo pathtracing}
\begin{itemize}
\item Теория распространения света
\item Модели материалов
\item Решение уравнение рендеринга
\item Оптимизация
\end{itemize}
\pause
\item \textbf{Лекции 7-8: Красивые картинки}
\begin{itemize}
\item Формат сцен glTF
\item Physically-based материалы
\item Тестуры
\item Material mapping
\end{itemize}
\end{itemize}
\end{frame}

\begin{frame}
\frametitle{Whitted-style raytracing (1979)}
\begin{figure}
\slideimage{stage1.jpg}
\end{figure}
\end{frame}

\againframe<2-3>{stages}

\begin{frame}
\frametitle{Monte-Carlo pathtracing}
\begin{figure}
\slideimage{stage2.jpg}
\end{figure}
\end{frame}

\againframe<3-4>{stages}

\begin{frame}
\frametitle{Красивые картинки}
\begin{figure}
\slideimage{stage3.jpg}
\end{figure}
\end{frame}

\begin{frame}<1-2>[label=what_is]
\frametitle{Что такое компьютерная графика?}
\begin{itemize}
\pause % 1
\item Кинематограф, мультипликация
\pause % 2
\item Компьютерные игры
\pause % 3
\item Рисунки, concept art
\pause % 4
\item Графический интерфейс
\pause % 5
\item Визуализация данных
\pause % 6
\item Научная визуализация
\pause % 7
\item Карты
\pause % 8
\item И т.д.
\end{itemize}
\end{frame}

\begin{frame}
\frametitle{The Matrix Revolutions (2003)}
\begin{figure}
\slideimage{matrix.jpg}
\end{figure}
\end{frame}

\begin{frame}
\frametitle{Avatar (2009)}
\begin{figure}
\slideimage{avatar.jpg}
\end{figure}
\end{frame}

\begin{frame}
\frametitle{The Avengers (2012)}
\begin{figure}
\slideimage{avengers.jpg}
\end{figure}
\end{frame}

\begin{frame}
\frametitle{Klaus (2019)}
\begin{figure}
\slideimage{klaus.jpg}
\end{figure}
\end{frame}

\againframe<3>{what_is}

\begin{frame}
\frametitle{Space Invaders (1978)}
\begin{figure}
\slideimage{space-invaders.jpg}
\end{figure}
\end{frame}

\begin{frame}
\frametitle{Doom (1993)}
\begin{figure}
\slideimage{doom.png}
\end{figure}
\end{frame}

\begin{frame}
\frametitle{Grand Theft Auto: Vice City (2002)}
\begin{figure}
\slideimage{vice-city.jpg}
\end{figure}
\end{frame}

\begin{frame}
\frametitle{Civilization V (2010)}
\begin{figure}
\slideimage{civ-5.jpg}
\end{figure}
\end{frame}

\begin{frame}
\frametitle{The Witcher 3: Wild Hunt (2015)}
\begin{figure}
\slideimage{witcher.jpg}
\end{figure}
\end{frame}

\begin{frame}
\frametitle{Cyberpunk 2077 (2020)}
\begin{figure}
\slideimage{cyberpunk.jpg}
\end{figure}
\end{frame}

\againframe<4>{what_is}

\begin{frame}
\begin{figure}
\slideimage{night.png}
\end{figure}
\end{frame}

\begin{frame}
\begin{figure}
\slideimage{tunnel.png}
\end{figure}
\end{frame}

\begin{frame}
\begin{figure}
\slideimage{forest.jpg}
\end{figure}
\end{frame}

\againframe<5>{what_is}

\begin{frame}
\frametitle{Mac OS Catalina}
\begin{figure}
\slideimage{macos.png}
\end{figure}
\end{frame}

\begin{frame}
\frametitle{Windows 10}
\begin{figure}
\slideimage{windows.png}
\end{figure}
\end{frame}

\begin{frame}
\frametitle{Europa Universalis 4}
\begin{figure}
\slideimage{eu4.png}
\end{figure}
\end{frame}

\begin{frame}
\frametitle{Dear ImGui}
\begin{figure}
\slideimage{imgui.png}
\end{figure}
\end{frame}

\againframe<6>{what_is}

\begin{frame}
\frametitle{Популярность браузеров в 2002-2009}
\begin{figure}
\slideimage{browsers.png}
\end{figure}
\end{frame}

\begin{frame}
\frametitle{Карта землетрясений}
\begin{figure}
\slideimage{earthquakes.png}
\end{figure}
\end{frame}

\begin{frame}
\frametitle{Количество случаев заражения COVID-19}
\begin{figure}
\slideimage{covid.png}
\end{figure}
\end{frame}

\againframe<7>{what_is}

\begin{frame}
\frametitle{Неустойчивость Рэлея — Тейлора}
\begin{figure}
\slideimage{fluids.jpg}
\end{figure}
\end{frame}

\begin{frame}
\frametitle{Молекулярные орбитали бензола}
\begin{figure}
\slideimage{benzene.png}
\end{figure}
\end{frame}

\begin{frame}
\frametitle{Симуляция напряжений в стенте методом конечных элементов}
\begin{figure}
\slideimage{stent.jpg}
\end{figure}
\end{frame}

\againframe<8>{what_is}

\begin{frame}
\frametitle{Схематическая карта}
\begin{figure}
\slideimage{map.png}
\end{figure}
\end{frame}

\begin{frame}
\frametitle{Спутниковая карта}
\begin{figure}
\slideimage{satellite.png}
\end{figure}
\end{frame}

\begin{frame}
\frametitle{Карта погоды}
\begin{figure}
\slideimage{weather.png}
\end{figure}
\end{frame}

\againframe<9>{what_is}

\begin{frame}<1-2>[label=classification]
\frametitle{Грубая и неточная классификация}
\pause
\begin{itemize}
\item \only<2>{2D / 3D} \pause \only<3->{2D / 2.5D / 3D}
\pause
\item Векторная / растровая
\pause
\item \only<5>{Realtime / offline} \pause \only<6->{Realtime / near real-time / offline}
\pause
\item Фотореалистичная / стилизованная
\pause
\item CPU / GPU
\end{itemize}
\end{frame}

\begin{frame}
\frametitle{Super Mario Bros. (1983) - 2D}
\begin{figure}
\slideimage{mario.jpg}
\end{figure}
\end{frame}

\begin{frame}
\frametitle{Red Dead Redemption 2 (2018) - 3D}
\begin{figure}
\slideimage{rdr2.jpg}
\end{figure}
\end{frame}

\againframe<2-3>{classification}

\begin{frame}
\frametitle{Civilization III (2001) - 2.5D}
\begin{figure}
\slideimage{civ3.png}
\end{figure}
\end{frame}

\againframe<3-4>{classification}

\begin{frame}
\frametitle{Векторная графика}
\begin{figure}
\slideimage{vector.jpg}
\end{figure}
\end{frame}

\begin{frame}
\frametitle{Растровая графика}
\begin{figure}
\slideimage{raster.png}
\end{figure}
\end{frame}

\againframe<4->{classification}

\begin{frame}
\frametitle{Чем мы будем заниматься?}
\begin{itemize}
\item \only<-1>{2D / 2.5D / 3D}\only<2->{{2D / 2.5D / \alert{\textbf{\underline{3D}}}}}
\item \only<-2>{Векторная / растровая}\only<3->{{\alert{\textbf{\underline{Векторная / растровая}}}}}
\item Realtime / near real-time / \only<-3>{offline}\only<4->{{\alert{\textbf{\underline{offline}}}}}
\item \only<-4>{Фотореалистичная}\only<5->{{\alert{\textbf{\underline{Фотореалистичная}}}}} / стилизованная
\item \only<-5>{CPU}\only<6->{{\alert{\textbf{\underline{CPU}}}}} / GPU
\end{itemize}
\end{frame}

\begin{frame}
\frametitle{Где это пригодится?}
\begin{itemize}
\pause
\item Разработка \textit{профессиональных движков фотореалистичного рендеринга}
\end{itemize}
\end{frame}

\begin{frame}
\frametitle{Чем мы \underline{не} будем заниматься?}
\begin{itemize}
\pause
\item \textbf{Учиться рисовать / моделировать и анимировать объекты / etc.}
\pause
\begin{itemize}
\item Красивая картинка -- движок + данные (текстуры, модели, частицы, etc, -- \textit{assets})
\item Курс про \alert{\textbf{\underline{\textit{движок}}}}
\end{itemize}
\end{itemize}
\end{frame}

\begin{frame}
\frametitle{Два способа рендеринга}
\begin{itemize}
\item \textit{Рендеринг} -- генерация изображения по входной сцене
\pause
\item Первый способ: \textit{растеризация}
\pause
\begin{itemize}
\item Разобьём сцену на максимально простые куски (треугольники)
\item Вычислим, в какие пиксели на экране попал каждый треугольник
\item Сделаем тонну трюков, чтобы обработать наложение объектов, освещение, и т.д.
\item {\color{green}\texttt{+}} Очень быстро работает
\item {\color{green}\texttt{+}} Требует только простейших операций
\item {\color{green}\texttt{+}} Хорошо ложится на GPU
\item {\color{red}\texttt{-}} Очень сложно реализовать нетривиальные эффекты
\end{itemize}
\end{itemize}
\end{frame}

\begin{frame}
\frametitle{Два способа рендеринга}
\begin{itemize}
\item Второй способ: \textit{трассировка лучей/путей}
\pause
\begin{itemize}
\item Разобьём сцену на объекты
\item Для каждого пикселя построим луч из камеры в этот пиксель
\item Вычислим, что пересёк этот луч
\item {\color{red}\texttt{-}} Работает медленнее из-за поиска пересечений
\item {\color{red}\texttt{-}} Плохо ложится на GPU из-за необходимости передачи сложных структур данных
\item {\color{green}\texttt{+}} С помощью дополнительных лучей легко реализовать почти любые эффекты, связанные с распространением света
\end{itemize}
\pause
\item Это то, чем мы будем заниматься
\end{itemize}
\end{frame}

\begin{frame}
\frametitle{Трассировка лучей}
Общий план рисования сцены трассировкой лучей:
\pause
\begin{itemize}
\item 1. Описать сцену: форму объектов и их материал (напр. цвет)
\pause
\item 2. Описать камеру: где она находится, куда смотрит
\pause
\item 3. Задать размеры изображения
\pause
\item 4. Для каждого пикселя изображения, построить луч из камеры в этот пиксель
\pause
\item 5. Найти ближайшее пересечение луча с объектами сцены
\pause
\item 6. В качестве цвета пикселя взять цвет объекта пересечения
\pause
\item 7. Сохранить полученную картинку
\end{itemize}
\end{frame}

\begin{frame}
\frametitle{Описание сцены}
\begin{itemize}
\item Поначалу мы будем использовать максимально простой формат сцен, описанный в слайдах практики
\pause
\item Ближе к концу курса перейдём к более серьёзному формату, скорее всего -- \texttt{glTF}
\end{itemize}
\end{frame}

\begin{frame}<1>[label=camera]
\frametitle{Описание камеры}
\begin{itemize}
\item Обычно в компьютерной графике используют два типа камеры (\textit{проекции}): \textit{ортографическую} и \textit{перспективную}
\pause
\item Мы будем использовать \textit{перспективную} камеру
\end{itemize}
\end{frame}

\begin{frame}
\frametitle{Ортографическая камера}
\begin{figure}
\slideimage{orthographic.png}
\end{figure}
\end{frame}

\begin{frame}
\frametitle{Перспективная камера}
\begin{figure}
\slideimage{perspective.png}
\end{figure}
\end{frame}

\againframe<1->{camera}

\begin{frame}[fragile]
\frametitle{Перспективная проекция}
\begin{itemize}
\item Есть {\color{blue}центр проекции} и {\color{blue}плоскость проекции}
\item \only<1>{\phantom{Проекция точки -- пересечение прямой, проходящей через эту точку и центр проекции, с плоскостью проекции}} \only<2->{{\color{red}Проекция} {\color{green}точки} -- пересечение {\color{magenta}прямой}, проходящей через эту {\color{green}точку} и {\color{blue}центр проекции}, с {\color{blue}плоскостью проекции}}
\end{itemize}
\begin{center}
\begin{tikzpicture}
\only<1->{
  \draw[blue,thick] (2.0, -2.0) -- (2.0, 2.0);
  \node[circle,draw=blue,fill=blue] at (0.0, 0.0) {};
}
\only<1->{
  \phantom{
    \draw[magenta,thick,dashed] (0.0, 0.0) -- (6.0, 1.5);
    \node[circle,draw=red,fill=red] at (2.0, 0.5) {};
    \node[circle,draw=green,fill=green] at (6.0, 1.5) {};
  }
}
\only<2->{
  \draw[magenta,thick,dashed] (0.0, 0.0) -- (6.0, 1.5);
  \node[circle,draw=red,fill=red] at (2.0, 0.5) {};
  \node[circle,draw=green,fill=green] at (6.0, 1.5) {};
}
\end{tikzpicture}
\end{center}
\end{frame}

\begin{frame}
\frametitle{Описание камеры}
\begin{itemize}
\item Позиция камеры: точка, из которой камера смотрит
\pause
\item Оси камеры (вперёд, вправо, вверх): векторы, задающие ориентацию камеры
\pause
\begin{itemize}
\item \alert{\textbf{N.B.}} Будем считать, что оси в таком порядке задают \textit{левую} систему координат
\end{itemize}
\pause
\item Угол обзора камеры по ширине и высоте (\textit{FOV -- field of view})
\end{itemize}
\end{frame}

\begin{frame}
\frametitle{Углы обзора камеры}
\begin{figure}
\slideimage{fov.png}
\end{figure}
\end{frame}

\begin{frame}
\frametitle{Углы обзора камеры}
\begin{itemize}
\item Угол обзора по ширине \begin{math}fov_X\end{math} -- угол между левой и правой границами видимой области
\pause
\item Угол обзора по высоте \begin{math}fov_Y\end{math} -- угол между верхней и нижней границами видимой области
\end{itemize}
\end{frame}

\begin{frame}
\frametitle{Вычисление луча из камеры}
\begin{itemize}
\item Хотим вычислить луч из камеры в пиксель с (целыми) координатами \begin{math}(P_X, P_Y)\end{math}
\pause
\item Луч задаётся началом и направлением
\pause
\item Начало луча \begin{math}O\end{math} это позиция камеры
\pause
\item Направление луча рассчитывается из координат пикселя, углов обзора камеры и её ориентации
\end{itemize}
\end{frame}

\begin{frame}
\frametitle{Вычисление луча из камеры}
\begin{itemize}
\item Игнорируя ориентацию камеры, представим, что она находится в начале координат и её оси совпадают с осями координат (камера смотрит в направлении оси Z)
\pause
\item Возьмём плоскость \begin{math}Z=1\end{math} -- какие точки этой плоскости видны камерой?
\pause
\item Точка \begin{math}(X, Y, 1)\end{math} видна, если \begin{math}|X| < \tan\left({fov_X\over 2}\right)\end{math} и \begin{math}|Y| < \tan\left({fov_Y\over 2}\right)\end{math}
\end{itemize}
\end{frame}

\begin{frame}
\frametitle{Вычисление луча из камеры}
\begin{itemize}
\item Значит, нам нужно перевести диапазон пиксельных координат в координаты точек на плоскости \begin{math}Z=1\end{math}:
\begin{itemize}
\item По X: \begin{math}[0, W]\rightarrow\left[-\tan\left({fov_X\over 2}\right), \tan\left({fov_X\over 2}\right)\right]\end{math}
\item По Y: \begin{math}[0, H]\rightarrow\left[-\tan\left({fov_Y\over 2}\right), \tan\left({fov_Y\over 2}\right)\right]\end{math}
\end{itemize}
\pause
\item Это делается формулами
\begin{itemize}
\item \begin{math}P_X \mapsto \left(2 \frac{P_X}{W} - 1\right) \cdot \tan\left({fov_X\over 2}\right)\end{math}
\item \begin{math}P_Y \mapsto \left(2 \frac{P_Y}{H} - 1\right) \cdot \tan\left({fov_Y\over 2}\right)\end{math}
\end{itemize}
\pause
\item \alert{\textbf{N.B.:}} Вместо точки \begin{math}(P_X, P_Y)\end{math}, соответствующей углу пикселя, лучше брать его центр \begin{math}(P_X + 0.5, P_Y + 0.5)\end{math}
\item \alert{\textbf{N.B.:}} Обычно форматы изображений хранят пиксели сверху-вниз, и картинка получится перевёрнутой; проще всего это исправить, домножив формулу для Y на -1
\end{itemize}
\end{frame}

\begin{frame}
\frametitle{Вычисление луча из камеры}
\begin{itemize}
\item Итак, по координатам пикселя \begin{math}(P_X, P_Y)\end{math} мы нашли вектор направления \begin{math}(X, Y, Z)\end{math} с \begin{math}Z=1\end{math}
\pause
\item Как его перевести в систему координат камеры? \pause Взять линейную комбинацию осей камеры с этими коэффициентами:
\begin{equation}
D = X\cdot\operatorname{right}+Y\cdot\operatorname{up}+Z\cdot\operatorname{forward}
\end{equation}
\pause
\item Это и есть финальный вектор направления для нашего луча
\pause
\item \alert{\textbf{N.B.:}} Обычно вектор направления считают нормированным -- это упрощает некоторые вычисления, но может усложнить другие; вы можете сделать так, как вам кажется удобнее
\end{itemize}
\end{frame}

\begin{frame}
\frametitle{Вычисление луча из камеры: aspect ratio}
\begin{itemize}
\item Отношение ширины к высоте \begin{math}W / H\end{math} изображения называется \textit{aspect ratio}
\pause
\item Чтобы изображение получилось без искажений, нужно, чтобы это отношение совпадало с \begin{math}\tan\left({fov_X\over 2}\right) / \tan\left({fov_Y\over 2}\right)\end{math}
\pause
\item В формате сцены мы будем задавать только \begin{math}fov_X\end{math}, а \begin{math}fov_Y\end{math} вычислять на основе остальных значений
\end{itemize}
\end{frame}

\begin{frame}
\frametitle{Вычисление пересечения луча и объектов}
\begin{itemize}
\item Как вычислить пересечение луча и объекта сцены? \pause Зависит от \textit{геометрии} объекта
\pause
\item Точки \begin{math}P = (X,Y,Z)\end{math} луча удобно параметризовать параметром \begin{math}t \in [0, \infty)\end{math} как \begin{math}P = O + t \cdot D\end{math}
\pause
\item Позже мы будем использовать настоящие трёхмерные модели из треугольников, а пока ограничимся простыми геометрическими фигурами: плоскостью, эллипсоидом, параллелепипедом
\pause
\item \alert{\textbf{N.B.:}} Пока будем считать, что все объекты находятся в начале координат
\end{itemize}
\end{frame}

\begin{frame}
\frametitle{Вычисление пересечения луча и плоскости}
\begin{itemize}
\item Плоскость удобно описывать вектором нормали и точкой, через которую проходит плоскость
\pause
\item Мы договорились, что все объекты находятся в начале координат, так что нам достаточно вектора нормали \begin{math}N = (N_X, N_Y, N_Z)\end{math}
\pause
\item Уравнение такой плоскости: \begin{math}P\cdot N = X\cdot N_X+Y\cdot N_Y+Z\cdot N_Z=0\end{math}
\pause
\item Подставляем формулу для точки на луче, и получаем \begin{math}O \cdot N + t \cdot D \cdot N = 0\end{math}, или \begin{math}t = -\frac{O \cdot N}{D \cdot N}\end{math}
\pause
\item Если \begin{math}t<0\end{math}, плоскость не видна в этом пикселе, иначе -- видна
\end{itemize}
\end{frame}

\begin{frame}
\frametitle{Вычисление пересечения луча и сферы}
\begin{itemize}
\item Сфера описывается радиусом \begin{math}R\end{math} и уравнением
\begin{equation}
X^2+Y^2+Z^2=P\cdot P = R^2
\end{equation}
\pause
\item Подставляем формулу точки луча, и получаем
\begin{equation}
O\cdot O + 2t (O\cdot D) + t^2 (D\cdot D) = R^2
\end{equation}
\pause
\item Это квадратное уравнение, у него может быть до двух решений:
\pause
\begin{itemize}
\item Если решений нет, луч не пересекает сферу
\item Если решение одно, луч касается сферы
\item Если решений два, луч пересекает сферу в двух точках и проходит через её внутренность
\end{itemize}
\pause
\item Нам нет нужды различать случаи одного и двух решений -- будем считать, что их всегда два
\end{itemize}
\end{frame}

\begin{frame}
\frametitle{Вычисление пересечения луча и сферы}
\begin{itemize}
\item Пусть \begin{math}t_1 < t_2\end{math} -- два решения квадратного уравнения
\pause
\item Могут произойти три вещи:
\pause
\begin{itemize}
\item \begin{math}0 < t_1\end{math} -- камера находится снаружи сферы, \begin{math}t_1\end{math} -- ближайшее пересечение
\pause
\item \begin{math}t_1 < 0 < t_2\end{math} -- камера находится внутри сферы, \begin{math}t_1\end{math} -- точка сзади камеры, \begin{math}t_2\end{math} -- ближайшее видимое пересечение
\pause
\item \begin{math}t_2 < 0\end{math} -- камера находится снаружи сферы, \begin{math}t_2\end{math} -- точка сзади камеры, нет видимого пересечения
\end{itemize}
\pause
\item То есть среди значений \begin{math}t_1, t_2\end{math} нас интересует минимальное, большее нуля; если таких нет, то сфера не видна этим лучом
\end{itemize}
\end{frame}

\begin{frame}
\frametitle{Вычисление пересечения луча и эллипсоида}
\begin{itemize}
\item Эллипсоид описывается тремя радиусами \begin{math}R=(R_X, R_Y, R_Z)\end{math} и уравнением
\begin{equation}
\frac{X^2}{R_X^2} + \frac{Y^2}{R_Y^2} + \frac{Z^2}{R_Z^2} = \frac{P}{R}\cdot \frac{P}{R} = 1
\end{equation}
\pause
\item Здесь, \begin{math}\frac{P}{R}\end{math} -- \textit{покомпонентное} деление векторов
\pause
\item Подставляем формулу точки луча, и получаем
\begin{equation}
\frac{O}{R}\cdot \frac{O}{R} + 2t \left(\frac{O}{R}\cdot \frac{D}{R}\right) + t^2 \left(\frac{D}{R}\cdot \frac{D}{R}\right) = 1
\end{equation}
\pause
\item Это квадратное уравнение, дальше всё аналогично ситуации со сферой
\end{itemize}
\end{frame}

\begin{frame}
\frametitle{Вычисление пересечения луча и параллелепипеда}
\begin{itemize}
\item Считая, что центр параллелепипеда находится в центре координат, удобно его описать размерами по трём осям \begin{math}S=(S_X, S_Y, S_Z)\end{math} и уравнениями внутренности
\begin{equation}
|X| < S_X
\end{equation}
\begin{equation}
|Y| < S_Y
\end{equation}
\begin{equation}
|Z| < S_Z
\end{equation}
\pause
\item Удобно представить параллелепипед как пересечение трёх бесконечных множеств \begin{math}|X| < S_X\end{math}, \begin{math}|Y| < S_Y\end{math} и \begin{math}|Z| < S_Z\end{math}
\pause
\item Подставляем формулу точки луча в уравнение границы по X: \begin{math}X = \pm S_X\end{math} и получаем
\begin{equation}
(O+t\cdot D)_X = \pm S_X \Longrightarrow t = \frac{\pm S_X - O_X}{D_X}
\end{equation}
\end{itemize}
\end{frame}

\begin{frame}
\frametitle{Вычисление пересечения луча и параллелепипеда}
\begin{itemize}
\item Пусть \begin{math}t_{1,X} < t_{2,X}\end{math} -- значения, полученные по формуле с предыдущего слайда
\pause
\item Аналогично \begin{math}t_{1,Y} < t_{2,Y}\end{math} и \begin{math}t_{1,Z} < t_{2,Z}\end{math}
\pause
\item Мы получили три диапазона значений \begin{math}t\end{math}, нас интересует их пересечение:
\begin{equation}
t_1 = \max(t_{1,X}, t_{1,Y}, t_{1,Z})
\end{equation}
\begin{equation}
t_2 = \min(t_{2,X}, t_{2,Y}, t_{2,Z})
\end{equation}
\pause
\item Если \begin{math}t_1 > t_2\end{math}, то пересечения нет
\pause
\item Иначе -- аналогично сфере, сравниваем значения с нулём
\end{itemize}
\end{frame}

\begin{frame}
\frametitle{Вычисление пересечения луча и сдвинутого объекта}
\begin{itemize}
\item Что, если объект находится не в начале координат, а в точке \begin{math}T\end{math}?
\pause
\item Переместим на вектор \begin{math}-T\end{math} и объект, и луч, т.е вычтем \begin{math}T\end{math} из точки начала луча  (объект переместим только мысленно)
\pause
\item Теперь объект в начале координат \begin{math}\Longrightarrow\end{math} используем описанные ранее алгоритмы
\end{itemize}
\end{frame}

\begin{frame}
\frametitle{Вычисление пересечения луча и повёрнутого объекта}
\begin{itemize}
\item Что, если объект не параллелен осям координат, а повёрнут?
\pause
\item Есть разные способы описания вращений: матрицы, углы Эйлера, кватернионы
\pause
\item Мы будем использовать кватернионы
\end{itemize}
\end{frame}

\begin{frame}
\frametitle{Кватернионы \begin{math}\mathbb{H}\end{math}: TL;DR}
\begin{itemize}
\item \begin{math}\mathbb{H}\end{math} -- четырёхмерная алгебра над вещественными числами \begin{math}\mathbb{R}\end{math}, т.е. описываются четыремя координатами \begin{math}(X,Y,Z,W)\end{math}
\pause
\item Можно складывать, умножать, делить (умножение не коммутативно)
\pause
\item Любое вращение трёхмерного пространства описывается двумя противоположными (\begin{math}Q_1 = -Q_2\end{math}) единичными (\begin{math}X^2+Y^2+Z^2+W^2=1\end{math}) кватернионами
\end{itemize}
\end{frame}

\begin{frame}
\frametitle{Кватернионы \begin{math}\mathbb{H}\end{math}: TL;DR}
\begin{itemize}
\item Кватернион \begin{math}(X,Y,Z,W)\end{math} удобно разбить на скалярную \begin{math}W\end{math} и векторную \begin{math}V = (X, Y, Z)\end{math} части
\pause
\item Умножение кватернионов в такой форме:
\begin{equation}
(V_1,W_1)\cdot(V_2,W_2) = (W_1\cdot V_2 + W_2\cdot V_1 + V_1\times V_2, W_1\cdot W_2 - V_1\cdot V_2)
\end{equation}
\pause
\item Сопряжённый кватернион:
\begin{equation}
\overline{(V,W)} = (-V, W)
\end{equation}
\end{itemize}
\end{frame}

\begin{frame}
\frametitle{Кватернионы и вращения}
\begin{itemize}
\item Любой единичный кватернион \begin{math}(V,W)\end{math} определяет вращение вектора \begin{math}X\end{math} по формуле
\begin{equation}
X \mapsto (V, W) \cdot (X, 0) \cdot (-V, W)
\end{equation}
\pause
\item Вращение на угол \begin{math}\theta\end{math} вокруг вектора \begin{math}V\end{math} задаёт кватернион
\begin{equation}
\left(\sin{\theta\over 2} \cdot V, \cos{\theta\over 2}\right)
\end{equation}
\pause
\item Сопряжённый кватернион описывает обратное вращение
\end{itemize}
\end{frame}

\begin{frame}
\frametitle{Вычисление пересечения луча и повёрнутого объекта}
\begin{itemize}
\item Что, если объект не параллелен осям координат, а повёрнут на кватернион \begin{math}Q\end{math}?
\pause
\item Повернём и объект, и вектор направления луча на сопряжённый кватернион \begin{math}\overline{Q}\end{math} (объект -- только мысленно)
\pause
\item Теперь объект параллелен осям координат \begin{math}\Longrightarrow\end{math} используем описанные ранее алгоритмы
\end{itemize}
\end{frame}

\begin{frame}
\frametitle{Вычисление пересечения луча и произвольного объекта}
\begin{itemize}
\item Что, если объект и повёрнут, и сдвинут?
\pause
\item Вычтем \begin{math}T\end{math} из точки начала луча
\pause
\item Повернём вектор направления луча на сопряжённый кватернион \begin{math}\overline{Q}\end{math}
\pause
\item Используем описанные ранее алгоритмы
\end{itemize}
\end{frame}

\begin{frame}
\frametitle{Вычисление пересечения луча и сцены}
\begin{itemize}
\item Проходим по всем объектам сцены, находим пересечение луча с ними
\pause
\item Из всех объектов, пересекших луч, находим ближайшее к камере пересечение, т.е. с минимальным значением \begin{math}t\end{math}
\pause
\item Если было хотя бы одно пересечение, записываем в пиксель цвет ближайшего объекта
\pause
\item Если пересечений не было, записываем в пиксель цвет фона
\end{itemize}
\end{frame}

\begin{frame}
\frametitle{Сохранение изображения}
\begin{itemize}
\item Большинство форматов изображений довольно сложны для того, чтобы вручную их записывать, -- для этого нужны сторонние библиотеки
\pause
\item Есть формат NetPBM (\texttt{.pbm/.pgm/.ppm}), ориентированный на простоту чтения и записи
\pause
\item Его поддерживают многие редакторы, и большинство просмотрщиков изображений
\pause
\item Мы будем использовать бинарную версию формата PPM (P6, его описание есть в слайдах практики)
\end{itemize}
\end{frame}

\begin{frame}
\frametitle{Сохранение изображения}
\begin{itemize}
\item Для записи изображения в файл нам потребуется массив его пикселей
\pause
\usemintedstyle{solarized-light}
\item Удобнее всего хранить пиксели как непрерывный массив байт (напр. \mintinline{cpp}|vector<uint8_t>| в C++), последовательно хранящий строки изображения
\pause
\item Пиксель -- три байта: его {\color{red}красный}, {\color{green}зелёный} и {\color{blue}синий} каналы, соответственно (в диапазоне [0..255])
\pause
\item Для доступа к i-ой компоненте пикселя (x,y) в непрерывном массиве можно использовать \mintinline{cpp}|pixels[(y * width + x) * 3 + i]|
\pause
\item Мы будем работать с цветами как с floating-point числами, как правило, в диапазоне [0..1]
\pause
\item После того, как цвет пикселя вычислен, его нужно будет сконвертировать в целое число [0..255]
\end{itemize}
\end{frame}

\begin{frame}
\frametitle{Литература, ссылки}
\begin{itemize}
\item Серия книжек \href{https://raytracing.github.io/}{\texttt{Raytracing In One Weekend}} -- очень хороший учебник начального уровня
\item \href{https://www.gabrielgambetta.com/computer-graphics-from-scratch/}{\texttt{Computer Graphics from Scratch}} -- книжка начального уровня
\item \href{https://www.pbr-book.org/}{\texttt{Physically Based Rendering: From Theory To Implementation}} -- книжка продвинутого уровня
\item \href{https://www.realtimerendering.com/raytracing.html}{\texttt{realtimerendering.com/raytracing.html}} -- сборник ресурсов про рейтрейсинг
\end{itemize}
\end{frame}

\end{document}