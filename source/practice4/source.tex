% (c) Nikita Lisitsa, lisyarus@gmail.com, 2024

\documentclass[10pt,handout]{beamer}

\usepackage[T2A]{fontenc}
\usepackage[russian]{babel}
\usepackage{minted}

\usepackage{graphicx}
\graphicspath{ {./images/} }

\usepackage{adjustbox}

\usepackage{color}
\usepackage{soul}

\usepackage{hyperref}

\usetheme{metropolis}

\definecolor{red}{rgb}{1,0,0}
\definecolor{green}{rgb}{0,0.5,0}
\definecolor{blue}{rgb}{0,0,1}
\definecolor{codebg}{RGB}{29,35,49}
\definecolor{lightbg}{RGB}{253,246,227}
\setminted{fontsize=\footnotesize}

\makeatletter
\newcommand{\slideimage}[1]{
  \begin{figure}
    \begin{adjustbox}{width=\textwidth, totalheight=\textheight-2\baselineskip-2\baselineskip,keepaspectratio}
      \includegraphics{#1}
    \end{adjustbox}
  \end{figure}
}
\makeatother

\title{Фотореалистичный рендеринг\quad\quad\quad\quad\quad\quad \textit{(aka raytracing)}}
\subtitle{Практика 4}
\date{2024}

\setbeamertemplate{footline}[frame number]

\begin{document}

\frame{\titlepage}

\begin{frame}[fragile]
\frametitle{Описание практики}
\begin{itemize}
\item Главная задача практики -- реализовать multiple importance sampling для диффузных поверхностей, как описано в лекции:
\begin{itemize}
\item Генерировать случайное направление cosine-weighted распределением или в сторону случайно выбранного ограниченного источника света (плоскости тут игнорируем)
\item Правильно вычислять суммарную вероятность этого семпла
\end{itemize}
\item Металлы и диэлектрики не трогаем, их код никак не меняется
\item Новых команд в формате сцены нет
\end{itemize}
\end{frame}

\begin{frame}[fragile]
\frametitle{Советы}
\begin{itemize}
\item МАКСИМАЛЬНО внимательно следим за всеми формулами и коэффициентами
\item Удобно под каждый тип распределения завести класс с такими методами:
\usemintedstyle{solarized-light}
\begin{itemize}
\item \mintinline{cpp}|vec3 sample(vec3 x, vec3 n)| -- генерирует случайный вектор направления из точки x с нормалью n
\item \mintinline{cpp}|float pdf(vec3 x, vec3 n, vec3 d)| -- вычисляет плотность вероятности направления d из точки x с нормалью n
\end{itemize}
\item Можно сделать \mintinline{cpp}|x| и \mintinline{cpp}|n| полями класса распределения, тогда они не нужны в аргументах методов
\end{itemize}
\end{frame}

\begin{frame}[fragile]
\frametitle{Советы}
\begin{itemize}
\item Напрашиваются такие классы распределений:
\usemintedstyle{solarized-light}
\begin{itemize}
\item \mintinline{cpp}|uniform| -- равномерное на полусфере (baseline для сравнения)
\item \mintinline{cpp}|cosine| -- cosine-weighted на полусфере
\item \mintinline{cpp}|light| -- направление на поверхность конкретного источника света
\item \mintinline{cpp}|mix(d1, d2, ...)| -- смесь нескольких распределений с одинаковыми вероятностями
\end{itemize}
\item Тогда наше распределение это \mintinline{cpp}|mix(cosine, mix(light(l0), light(l1), ...))|
\item Если нет ни одного источника, то будем использовать просто \mintinline{cpp}|cosine|
\end{itemize}
\end{frame}

\begin{frame}[fragile]
\frametitle{Советы}
\begin{itemize}
\item \mintinline{cpp}|cosine.pdf| должна возвращать 0, если \mintinline{cpp}|dot(n, d) < 0|
\item \mintinline{cpp}|light.pdf| должна возвращать 0, если луч \mintinline{cpp}|(x,d)| не пересекает источник света
\item В противном случае, \mintinline{cpp}|light.pdf| должна возвращать сумму по двум точкам пересечения
\item Обратите внимание, что в формуле перевода плотности семпла на поверхности источника в плотность семпла направления фигурирует косинус угла \textit{на поверхности источника}, а не косинус угла в освещаемой точке!
\end{itemize}
\end{frame}

\begin{frame}[fragile]
\frametitle{Примеры сцены}
Сцены из 3ей практики нам отлично подойдут:
\href{https://github.com/lisyarus/raytracing-course-slides/tree/trunk/example_scenes/practice3_1.txt}{\texttt{slides/tree/trunk/example\_scenes/practice3\_1.txt}}
\href{https://github.com/lisyarus/raytracing-course-slides/tree/trunk/example_scenes/practice3_2.txt}{\texttt{slides/tree/trunk/example\_scenes/practice3\_2.txt}}
\href{https://github.com/lisyarus/raytracing-course-slides/tree/trunk/example_scenes/practice3_3.txt}{\texttt{slides/tree/trunk/example\_scenes/practice3\_3.txt}}
\href{https://github.com/lisyarus/raytracing-course-slides/tree/trunk/example_scenes/practice3_4.txt}{\texttt{slides/tree/trunk/example\_scenes/practice3\_4.txt}}
\href{https://github.com/lisyarus/raytracing-course-slides/tree/trunk/example_scenes/practice3_5.txt}{\texttt{slides/tree/trunk/example\_scenes/practice3\_5.txt}}
\end{frame}

\end{document}
