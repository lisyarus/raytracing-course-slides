% (c) Nikita Lisitsa, lisyarus@gmail.com, 2024

\documentclass[10pt,handout]{beamer}

\usepackage[T2A]{fontenc}
\usepackage[russian]{babel}
\usepackage{minted}

\usepackage{graphicx}
\graphicspath{ {./images/} }

\usepackage{adjustbox}

\usepackage{color}
\usepackage{soul}

\usepackage{hyperref}

\usetheme{metropolis}

\definecolor{red}{rgb}{1,0,0}
\definecolor{green}{rgb}{0,0.5,0}
\definecolor{blue}{rgb}{0,0,1}
\definecolor{codebg}{RGB}{29,35,49}
\definecolor{lightbg}{RGB}{253,246,227}
\setminted{fontsize=\footnotesize}

\makeatletter
\newcommand{\slideimage}[1]{
  \begin{figure}
    \begin{adjustbox}{width=\textwidth, totalheight=\textheight-2\baselineskip-2\baselineskip,keepaspectratio}
      \includegraphics{#1}
    \end{adjustbox}
  \end{figure}
}
\makeatother

\title{Фотореалистичный рендеринг\quad\quad\quad\quad\quad\quad \textit{(aka raytracing)}}
\subtitle{Практика 3}
\date{2024}

\setbeamertemplate{footline}[frame number]

\begin{document}

\frame{\titlepage}

\begin{frame}[fragile]
\frametitle{Описание практики}
В этой практике:
\begin{itemize}
\item Источники света (в т.ч. ambient) становятся не нужны -- вместо них любой объект может излучать свет
\item Вычисление цвета пикселя нужно сделать с помощью Монте-Карло интегрирования
\item Описание материалов -- как в прошлой практике (но добавляется излучение)
\item Должно быть реализовано сглаживание, как описано в лекции
\item \textbf{\alert{N.B.}}: Всё усреднение делается \textbf{до} tone-mapping'а и гамма-коррекции!
\end{itemize}
\end{frame}

\begin{frame}
\frametitle{Формат сцены: новые команды}
\begin{itemize}
\item \texttt{SAMPLES <samples>} -- количество семплов на пиксель (один семпл здесь это один луч и все сделанные им рекурсивные вызовы)
\item \texttt{EMISSION <red> <green> <blue>} -- (при описании материала объекта) цвет излучения \begin{math}E\end{math} объекта, т.е. \begin{math}L_e(\omega) = E\end{math}
\end{itemize}
\end{frame}

\begin{frame}[fragile]
\frametitle{Примеры сцены}
\href{https://github.com/lisyarus/raytracing-course-slides/tree/trunk/example_scenes/practice3_1.txt}{\texttt{slides/tree/trunk/example\_scenes/practice3\_1.txt}}
\href{https://github.com/lisyarus/raytracing-course-slides/tree/trunk/example_scenes/practice3_2.txt}{\texttt{slides/tree/trunk/example\_scenes/practice3\_2.txt}}
\href{https://github.com/lisyarus/raytracing-course-slides/tree/trunk/example_scenes/practice3_3.txt}{\texttt{slides/tree/trunk/example\_scenes/practice3\_3.txt}}
\href{https://github.com/lisyarus/raytracing-course-slides/tree/trunk/example_scenes/practice3_4.txt}{\texttt{slides/tree/trunk/example\_scenes/practice3\_4.txt}}
\href{https://github.com/lisyarus/raytracing-course-slides/tree/trunk/example_scenes/practice3_5.txt}{\texttt{slides/tree/trunk/example\_scenes/practice3\_5.txt}}
\end{frame}

\begin{frame}
\frametitle{Пример сцены №1}
\begin{figure}
\slideimage{practice3_1.png}
\end{figure}
\end{frame}

\begin{frame}
\frametitle{Пример сцены №2}
\begin{figure}
\slideimage{practice3_2.png}
\end{figure}
\end{frame}

\begin{frame}
\frametitle{Пример сцены №3}
\begin{figure}
\slideimage{practice3_3.png}
\end{figure}
\end{frame}

\begin{frame}
\frametitle{Пример сцены №4}
\begin{figure}
\slideimage{practice3_4.png}
\end{figure}
\end{frame}

\begin{frame}
\frametitle{Пример сцены №5}
\begin{figure}
\slideimage{practice3_5.png}
\end{figure}
\end{frame}

\begin{frame}
\frametitle{Примеры сцен}
\begin{itemize}
\item Изображения тестовых сцен сгенерированы с довольно большим числом семплов (\begin{math}\sim\end{math}4k)
\item При ручном тестировании удобнее использовать маленькое число семплов -- скажем, 64 или даже 16
\item Придётся научиться глазами делать денойзинг :)
\end{itemize}
\end{frame}

\end{document}
