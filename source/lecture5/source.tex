% (c) Nikita Lisitsa, lisyarus@gmail.com, 2024

\documentclass[10pt]{beamer}

\usepackage[T2A]{fontenc}
\usepackage[russian]{babel}
\usepackage{minted}

\usepackage{graphicx}
\graphicspath{ {./images/} }

\usepackage{adjustbox}

\usepackage{color}
\usepackage{soul}

\usepackage{hyperref}

\usetheme{metropolis}

\definecolor{red}{rgb}{1,0,0}
\definecolor{green}{rgb}{0,0.5,0}
\definecolor{blue}{rgb}{0,0,1}
\definecolor{codebg}{RGB}{29,35,49}
\definecolor{lightbg}{RGB}{253,246,227}
\setminted{fontsize=\footnotesize}

\makeatletter
\newcommand{\slideimage}[1]{
  \begin{figure}
    \begin{adjustbox}{width=\textwidth, totalheight=\textheight-2\baselineskip-2\baselineskip,keepaspectratio}
      \includegraphics{#1}
    \end{adjustbox}
  \end{figure}
}
\makeatother

\title{Фотореалистичный рендеринг\quad\quad\quad\quad\quad\quad \textit{(aka raytracing)}}
\subtitle{Лекция 5: Оптимизация рейтрейсинга, распараллеливание, BVH}
\date{2024}

\setbeamertemplate{footline}[frame number]

\begin{document}

\frame{\titlepage}

\begin{frame}
\frametitle{Треугольники}
\begin{itemize}
\item В современности, основной примитив для описания трёхмерных объектов -- \textit{треугольник}
\pause
\item Почему?
\pause
\begin{itemize}
\item Простота описания
\pause
\item Простота рисования (важнее при растеризации)
\pause
\item Простота интерполяции значений вершин
\pause
\item Возможность аппроксимировать любые фигуры
\end{itemize}
\end{itemize}
\end{frame}

\begin{frame}
\frametitle{Треугольники}
\begin{itemize}
\item Как поддержать треугольник в нашем рейтрейсере?
\pause
\item Нам нужны три операции:
\pause
\begin{itemize}
\item Вычислить пересечение луча и треугольника
\pause
\item Вычислить нормаль к треугольнику
\pause
\item Сгенерировать случайную точку на треугольнике
\end{itemize}
\end{itemize}
\end{frame}

\begin{frame}
\frametitle{Треугольники}
\begin{itemize}
\item Точки треугольника с вершинами \begin{math}(A,B,C)\end{math} удобно параметризовать двумя параметрами \begin{math}(u,v)\end{math} как
\begin{equation*}
A + u\cdot (B-A) + v\cdot (C-A)
\end{equation*}
\pause
\item Точка принадлежит треугольнику, если \begin{math}0 \leq u, 0 \leq v, u + v \leq 1\end{math}
\pause
\item Произвольные значения \begin{math}(u,v)\end{math} описывают точку на плоскости, в которой лежит треугольник
\pause
\item Тройка \begin{math}(1-u-v, u, v)\end{math} -- \textit{барицентрические координаты} точки относительно вершин треугольника
\end{itemize}
\end{frame}

\begin{frame}
\frametitle{Пересечение луча с треугольником}
\begin{itemize}
\item Пересечение луча \begin{math}O + t\cdot D\end{math} сводится к решению уравнения
\begin{equation*}
A + u\cdot (B-A) + v\cdot (C-A) = O + t\cdot D
\end{equation*}
\pause
\item Немного его перепишем:
\begin{equation*}
u\cdot (B-A) + v\cdot (C-A) - t\cdot D = O - A
\end{equation*}
\pause
\item Это линейная система уравнений \begin{math}3\times 3\end{math} с неизвестными \begin{math}(u,v,t)\end{math} и матрицей со столбцами \begin{math}(B-A, C-A, -D)\end{math}
\pause
\item Решить её можно явным обращением матрицы
\end{itemize}
\end{frame}

\begin{frame}
\frametitle{Пересечение луча с треугольником}
\begin{itemize}
\item Мы получили значения \begin{math}(u,v,t)\end{math} -- они описывают точку пересечения луча с плоскостью, содержащей треугольник
\pause
\item Если не выполняется хотя бы одно из условий \begin{math}0\leq u, 0\leq v, u+v \leq 1\end{math}, то точка не лежит в треугольнике \begin{math}\Longrightarrow\end{math} пересечения нет
\pause
\item Если \begin{math}t < 0\end{math}, то точка лежит сзади начала луча \begin{math}\Longrightarrow\end{math} пересечения нет
\pause
\item В противном случае пересечение есть
\end{itemize}
\end{frame}

\begin{frame}
\frametitle{Пересечение луча с треугольником}
\begin{itemize}
\item Нормаль к поверхности треугольника \begin{math}(A,B,C)\end{math} -- нормированный вектор \begin{math}N = (B-A)\times(C-A)\end{math}
\pause
\item Треугольник не имеет внутренности, поэтому будем считать вектор \begin{math}N\end{math} выходящим из внешней стороны треугольника, а вектор \begin{math}(-N)\end{math} -- из внутренней стороны
\pause
\item Как обычно, по знаку \begin{math}N\cdot D\end{math} понимаем, снаружи луч, или внутри, и переворачиваем нормаль, если луч внутри
\end{itemize}
\end{frame}

\begin{frame}
\frametitle{Семплинг поверхности треугольника}
\begin{itemize}
\item Так как преобразование \begin{math}(u, v) \mapsto P\end{math} -- аффинное, т.е. имеет константный Якобиан, нам достаточно сгенерировать равномерно точку в треугольнике \begin{math}0\leq u, 0\leq v, u+v \leq 1\end{math} -- соответствующая точка в треугольнике \begin{math}(A,B,C)\end{math} автоматически будет равномерно распределена
\pause
\item Генерируем равномерные \begin{math}U,V \sim U(0,1)\end{math}
\pause
\item Если \begin{math}U+V>1\end{math}, отражаем их через центр квадрата \begin{math}[0,1]\times [0,1]\end{math} преобразованием \begin{math}U \mapsto 1-U, V \mapsto 1-V\end{math}
\pause
\item Плотность полученной точки на треугольнике равна \begin{math}1/S\end{math}, где \begin{math}S = \frac{1}{2}\| (B-A)\times (C-A)\|\end{math} -- площадь треугольника
\end{itemize}
\end{frame}

\begin{frame}
\frametitle{Об оптимизации рендеринга}
\begin{itemize}
\item Мы видели, что Монте-Карло интегрирование занимает очень много времени, если мы хотим получить качественную картинку
\pause
\item Хочется научиться оптимизировать процесс
\pause
\item Оптимизировать = получить картинку более высокого качества за то же время
\pause
\item Два варианта:
\pause
\begin{itemize}
\item Увеличивать качество каждого луча (прошлая лекция)
\pause
\item Ускорять вычисление каждого луча (эта лекция)
\end{itemize}
\end{itemize}
\end{frame}

\begin{frame}[fragile]
\frametitle{Оптимизация -- это сложно}
На производительность CPU влияют:
\pause
\begin{itemize}
\item Общая загруженность системы
\pause
\item Количество и паттерн доступов к памяти (cache-friendliness)
\pause
\item Как функции программы лежат в памяти (опять кэш)
\pause
\item Branch prediction
\pause
\item Оптимизации компилятора
\pause
\item Throttling
\pause
\item Многое другое
\end{itemize}
\end{frame}

\begin{frame}[fragile]
\frametitle{Хорошие новости}
\begin{itemize}
\item Рейтрейсинг -- очень предсказуемый алгоритм: мы буквально огромное количество раз выполняем одну и ту же операцию
\pause
\item Как оптимизировать:
\pause
\begin{itemize}
\item Точечно оптимизировать конкретные участки кода
\pause
\item Ускорять алгоритмически
\pause
\item Распараллеливать
\end{itemize}
\end{itemize}
\end{frame}

\begin{frame}[fragile]
\frametitle{Точечные оптимизации}
\begin{itemize}
\item Более эффективная реализация конкретных операций
\pause
\begin{itemize}
\item Быстрое вращение вектора кватернионом \href{https://blog.molecular-matters.com/2013/05/24/a-faster-quaternion-vector-multiplication/}{\texttt{blog.molecular-matters.com/2013/05/24/a-faster-quaternion-vector-multiplication}}
\end{itemize}
\pause
\item Кеширование значений
\pause
\begin{itemize}
\item Нормаль и площадь треугольника
\pause
\item Вектор \begin{math}\left(\frac{1}{D_X}, \frac{1}{D_Y}, \frac{1}{D_Z}\right)\end{math}, использующийся при пересечения луча и параллелепипеда
\end{itemize}
\pause
\item Избегание бесполезных операций
\pause
\begin{itemize}
\item Не посылаем рекурсивно луч, если его вклад потом домножится на 0 (например, при importance sampling'е источника света с внутренней стороны поверхности объекта)
\end{itemize}
\end{itemize}
\end{frame}

\begin{frame}[fragile]
\frametitle{Точечные оптимизации}
\begin{itemize}
\item Компактное и непрерывное расположение данных в памяти
\pause
\begin{itemize}
\item Меньше указателей и pointer chasing'а
\pause
\item Больше простых структур в непрерывных массивах
\end{itemize}
\pause
\item Branch prediction
\pause
\begin{itemize}
\item Меньше виртуальных функций и иерархий классов
\pause
\item Замена вычислений branchless эквивалентами
\end{itemize}
\end{itemize}
\end{frame}

\begin{frame}[fragile]
\frametitle{Точечные оптимизации}
\begin{itemize}
\item Избегание лишних аллокаций памяти
\pause
\begin{itemize}
\item В основном коде рейтрейсера (после того, как сцена прочитана) их в идеале не должно быть вообще
\end{itemize}
\pause
\item SSE-реализации математических операций
\pause
\item См. \href{https://jacco.ompf2.com/2022/04/18/how-to-build-a-bvh-part-2-faster-rays/}{\texttt{jacco.ompf2.com/2022/04/18/how-to-build-a-bvh-part-2-faster-rays}}
\end{itemize}
\end{frame}

\begin{frame}[fragile]
\frametitle{Большие сцены}
\begin{itemize}
\item Точечные оптимизации могут дать вам весомое ускорение, вплоть до нескольких десятков процентов
\pause
\item Тем не менее, они никак не помогут нам с рендерингом больших сцен
\pause
\item В основе вычисления цвета луча -- алгоритм со сложностью \begin{math}N \cdot D\end{math}, где \begin{math}N\end{math} -- количество объектов в сцене, а \begin{math}D\end{math} -- максимальная глубина
\pause
\item Зависимость от \begin{math}N\end{math} \textit{линейная} -- это слишком медленно, если мы хотим рисовать сцены с миллионами треугольников!
\end{itemize}
\end{frame}

\begin{frame}[fragile]
\frametitle{Деревья}
\begin{itemize}
\item В вычислительной геометрии есть множество структур данных, позволяющих ускорить запросы к геометрическим данным, использующие их локальность
\pause
\item Почти всегда это какие-нибудь \textit{деревья}: R/R+/R*-деревья, квадро/октодеревья, BSP-деревья, \begin{math}\dots\end{math}
\pause
\item Общий термин для подобных древовидных пространственных структур данных -- \textit{bounding volume hierarchy (BVH)}
\pause
\item В рейтрейсинге под BVH обычно понимают разновидность R-деревьев
\end{itemize}
\end{frame}

\begin{frame}[fragile]
\frametitle{BVH: описание}
\begin{itemize}
\item Каждая вершина дерева описывает некоторую область пространства, ограниченную прямоугольным параллелепипедом, параллельным осям координат (\textit{axis-aligned bounding box, AABB})
\pause
\item Такой параллелепипед можно описать двумя точками -- с минимальными и максимальными координатами по X, Y и Z соответственно
\pause
\item У вершины могут быть дочерние вершины, AABB которых полностью содержатся в AABB самой вершины
\pause
\item У вершины может быть список объектов сцены, полностью содержащихся в AABB самой вершины
\pause
\item Все объекты сцены, кроме бесконечных (плоскостей), должны лежать в какой-то одной вершине дерева
\pause
\item \textbf{\alert{N.B.:}} AABB двух дочерних вершин одной вершины \underline{могут пересекаться}!
\end{itemize}
\end{frame}

\begin{frame}[fragile]
\frametitle{О хранении структур данных}
\begin{itemize}
\item Кажется естественным хранить дерево в виде указателя на корневую вершину, и в вершине хранить детей тоже в виде указателей
\pause
\item Это \textbf{плохой способ} по ряду причин:
\pause
\begin{itemize}
\item Создание дерева использует много маленьких аллокаций памяти
\pause
\item Вершины дерева будут лежать в памяти как попало, мало шансов иметь их близко в памяти
\pause
\item Сложно поддержать циклические связи между вершинами
\pause
\item Удаление дерева -- медленная рекурсивная операция
\pause
\item Копирование дерева (deep copy) -- медленная рекурсивная операция
\pause
\item Сериализация дерева -- медленная рекурсивная операция
\end{itemize}
\end{itemize}
\end{frame}

\begin{frame}[fragile]
\frametitle{О хранении структур данных}
\usemintedstyle{solarized-light}
\begin{itemize}
\item Есть способ лучше:
\pause
\begin{itemize}
\item Хранить все вершины дерева как структуры в непрерывном массиве (\mintinline{cpp}|std::vector|)
\pause
\item Использовать индексы этого массива как ссылки между вершинами
\end{itemize}
\pause
\item При таком способе:
\pause
\begin{itemize}
\item Создание дерева использует минимум аллокаций памяти
\pause
\item Вершины дерева будут лежать в памяти непрерывно
\pause
\item Циклические связи между вершинами тривиальны -- просто храним индексы
\pause
\item Удаление дерева сводится к одной деаллокации массива вершин
\pause
\item Копирование дерева (deep copy) работает автоматически (скопированные индексы будут ссылаться на скопированный массив вершин)
\pause
\item Сериализация дерева сводится к побайтовой сериализации массива вершин
\end{itemize}
\end{itemize}
\end{frame}

\begin{frame}[fragile]
\frametitle{BVH: хранение объектов сцены}
\begin{itemize}
\item Не нужно хранить объекты сцены прямо в вершинах, это сделает вершины дерева тяжёлыми и неэффективными
\pause
\item При построении дерева можно отсортировать массив объектов сцены так, чтобы объекты, лежащие в конкретной вершине, хранились непрерывно в массиве всех объектов сцены
\pause
\item Тогда в вершине дерева можно запомнить только индекс первого объекта сцены, лежащего в этой вершине, и количество таких объектов
\end{itemize}
\end{frame}

\begin{frame}[fragile]
\frametitle{BVH: описание в коде}
\usemintedstyle{solarized-light}
\begin{minted}[bgcolor=lightbg]{cpp}
struct Node
{
    vec3 aabb_min;
    vec3 aabb_max;
    uint32_t left_child;
    uint32_t right_child;
    uint32_t first_primitive_id;
    uint32_t primitive_count;
};

struct BVH
{
    vector<Node> nodes;
    uint32_t root;
};

struct Scene
{
    vector<Primitive> primitives;
    BVH bvh;
};
\end{minted}
\end{frame}

\begin{frame}[fragile]
\frametitle{Data-oriented design}
\begin{itemize}
\item \textit{DOD (data-oriented design)} (иногда \textit{DDD, data-driven design}) -- методология, призванная структурировать код и алгоритмы вокруг \textit{потока данных} вашей программы
\pause
\item Часто противопоставляется ООП по историческим причинам: 15 лет назад все всё писали на иерархиях классов, теперь всё пишут на непрерывных массивах
\pause
\item DOD -- очень полезная концепция, но, как и с любым другим инструментом, нужно понимать, где и зачем её применять
\end{itemize}
\end{frame}

\begin{frame}[fragile]
\frametitle{BVH: обход}
\begin{itemize}
\item Хотим найти пересечение луча со сценой
\pause
\item Рекурсивно спускаемся по дереву от корневой вершины
\pause
\item Если луч не пересекает AABB вершины, игнорируем эту вершину
\pause
\item В противном случае, пересекаем луч с объектами сцены в этой вершине (если они есть) и спускаемся в дочерние вершины (если они есть)
\end{itemize}
\end{frame}

\begin{frame}[fragile]
\frametitle{BVH: построение}
\usemintedstyle{solarized-light}
\begin{itemize}
\item Можно выделить рекурсивный метод
\begin{minted}{cpp}
BVH::build_node(vector<Primitive> & primitives,
    uint32_t first, uint32_t count)
\end{minted}
\pause
\item Тогда построение дерева сводится к вызову
\mintinline{cpp}|bvh.root = bvh.build_node(primitives, 0, primitives.size())|
\pause
\item \textbf{\alert{N.B.:}} Вместо индексов может быть удобно использовать итераторы
\end{itemize}
\end{frame}

\begin{frame}[fragile]
\frametitle{BVH: построение}
\usemintedstyle{solarized-light}
\begin{itemize}
\item При построении вершины дерева, надо определиться
\pause
\begin{itemize}
\item 1. Хотим ли мы её разбивать на дочерние вершины
\pause
\item 2. Если да, то как именно это сделать
\end{itemize}
\pause
\item Если \mintinline{cpp}|count| достаточно мал (например, не больше 4), то разбивать вершину дальше нет смысла -- мы быстрее проверим пересечение с самими объектами сцены, чем разберёмся с AABB вершины
\end{itemize}
\end{frame}

\begin{frame}[fragile]
\frametitle{BVH: построение}
\usemintedstyle{solarized-light}
\begin{itemize}
\item Если \mintinline{cpp}|count| больше порогового значения, надо как-то разбить объекты на две группы
\pause
\item Простой вариант -- разбить вдоль середины самого длинного ребра AABB вершины:
\pause
\begin{itemize}
\item Вычисляем, какое из рёбер (X, Y или Z) самое длинное
\pause
\item Разбиваем объекты \mintinline{cpp}|[first, first + count)| на две группы (в C++ это \mintinline{cpp}|std::partition|)
\pause
\item \mintinline{cpp}|[first, first + middle)| -- те, кто левее центра ребра (вдоль выбранной оси)
\item \mintinline{cpp}|[first + middle, first + count)| -- те, кто правее
\pause
\item Рекурсивно вызываем создание двух дочерних вершин на двух подмножествах объектов сцены
\end{itemize}
\end{itemize}
\end{frame}

\begin{frame}[fragile]
\frametitle{BVH: построение}
\usemintedstyle{solarized-light}
\begin{itemize}
\item При рекурсивном разбиении нужно аккуратно обработать патологические случаи
\pause
\item Если в одну из дочерних вершин попадает 0 объектов сцены, разбиение нужно отменить, и положить все объекты в текущую вершину
\pause
\item Иначе может произойти бесконечная рекурсия
\end{itemize}
\end{frame}

\begin{frame}[fragile]
\frametitle{BVH: вычисление AABB}
\begin{itemize}
\item Как вычислить AABB для примитивов?
\pause
\item Параллелепипед -- совпадает со своим AABB
\pause
\item Эллипсоид -- берём его радиусы, они описывают нужный AABB
\pause
\item Треугольник -- вычисляем минимальные и макимальные значения координат по всем трём осям
\end{itemize}
\end{frame}

\begin{frame}[fragile]
\frametitle{BVH: вычисление AABB}
\begin{itemize}
\item Если примитив сдвинут, то сдвиг нужно применить и к AABB
\pause
\item Если примитив повёрнут, то нужно вычислить повёрнутый AABB -- например, как bounding box восьми повёрнутых вершин не повёрнутого AABB
\pause
\item \textbf{\alert{N.B.:}} Как всегда, нужно сначала применить поворот, а потом сдвиг!
\end{itemize}
\end{frame}

\begin{frame}[fragile]
\frametitle{AABB: описание в коде}
\usemintedstyle{solarized-light}
\begin{minted}[bgcolor=lightbg]{cpp}
struct AABB
{
    glm::vec3 min;
    glm::vec3 max;

    void extend(vec3 p)
    {
        min = glm::min(min, p);
        max = glm::max(max, p);
    }

    void extend(AABB aabb)
    {
        min = glm::min(min, aabb.min);
        max = glm::max(max, aabb.max);
    }
};
\end{minted}
\end{frame}

\begin{frame}[fragile]
\frametitle{Ускорение обхода: early-out}
\begin{itemize}
\item Запоминаем расстояние до ближайшего уже найденного пересечения
\pause
\item При посещении вершины, если луч пересекает её AABB, вычисляем расстояние до ближайшего пересечения AABB вершины
\pause
\item Если расстояние до AABB вершины больше, чем ближайшее уже найденное пересечение с каким-то объектом, то вершину можно пропустить
\end{itemize}
\end{frame}

\begin{frame}[fragile]
\frametitle{Ускорение обхода: ordered traversal}
\begin{itemize}
\item При посещении дочерних вершин какой-либо вершины, мы можем выбрать оптимальный порядок их посещения
\pause
\item Например, если вершина была разбита по оси X, и при этом X-компонента вектора направления луча отрицательна, то имеет смысл сначала посетить правую вершину, и только потом левую
\pause
\item Для этого в вершинах нужно помнить, по какой оси было разбиение
\end{itemize}
\end{frame}

\begin{frame}[fragile]
\frametitle{BVH: улучшение качества дерева}
\begin{itemize}
\item Уже разбиение по центру самого длинного ребра даст хорошую производительность на типичных сценах, но мы можем сделать лучше
\pause
\item Пусть, мы придумали, как разбить конкретную вершину на две
\pause
\item Как оценить качество такого разбиения?
\end{itemize}
\end{frame}

\begin{frame}[fragile]
\frametitle{BVH: улучшение качества дерева}
\begin{itemize}
\item Пусть есть случайный луч, пересекающий родительскую вершину
\pause
\item Пусть \begin{math}p_{1,2}\end{math} -- вероятность, что такой луч попадёт в первую (вторую) дочернюю вершину
\pause
\item Пусть \begin{math}N_{1,2}\end{math} -- количество объектов в первой (второй) дочерней вершиной
\pause
\item Тогда в среднем нам придётся обработать \begin{math}p_1N_1+p_2N_2\end{math} объектов
\end{itemize}
\end{frame}

\begin{frame}[fragile]
\frametitle{BVH: SAH}
\begin{itemize}
\item Можно показать, что \begin{math}p_i = \frac{S_i}{S}\end{math}, где \begin{math}S_i\end{math} -- площадь поверхности AABB дочерней вершины, а \begin{math}S\end{math} -- площадь поверхности AABB родительской вершины
\pause
\item Тогда мы можем использовать величину \begin{math}S \cdot N\end{math} как \textit{стоимость} вершины
\pause
\item Разбиение вершины на дочерние имеет смысл, если стоимость вершины больше, чем сумма стоимости дочерних вершин
\pause
\item Этот подход называется \textit{surface area heuristic (SAH)}
\end{itemize}
\end{frame}

\begin{frame}[fragile]
\frametitle{BVH: SAH}
\begin{itemize}
\item Рассмотрим идеальный случай: разбиение по центру самого длинного ребра даёт две непересекающихся дочерних вершины, не пересекающих плоскость разбиения
\pause
\item Тогда площадь поверхности AABB дочерних вершин \begin{math}S_{1,2} \leq \frac{2}{3}S\end{math}
\pause
\item И суммарная стоимость дочерних вершин: \begin{math}S_1N_1+S_2N_2 \leq \frac{2}{3}S\frac{1}{2}N_1+\frac{2}{3}S\frac{1}{2}(N-N_1) = \frac{2}{3}SN\end{math}
\pause
\item \begin{math}\Longrightarrow\end{math} Стоимость разбиения меньше стоимости родительской вершины
\end{itemize}
\end{frame}

\begin{frame}[fragile]
\frametitle{BVH: SAH}
\begin{itemize}
\item Мы научились измерять качество конкретного разбиения
\pause
\item Как найти идеальное разбиение? \pause Перебором!
\pause
\item Переберём все 3 возможных оси разбиения
\pause
\item Отсортируем объекты данной вершины в порядке возрастания i-ой координаты центра объекта (или его AABB), где i -- выбранная ось разбиение
\pause
\item Для каждой оси у нас есть \begin{math}N-1\end{math} вариантов разбиения (первые \begin{math}K\end{math} объектов идут в левую вершину, оставшиеся \begin{math}N-K\end{math} -- в правую)
\pause
\item Вычислим стоимость разбиения для всех этих вариантов и возьмём минимальный
\pause
\item Если стоимость минимального разбиения всё ещё выше стоимости самой вершины, не производим разбиение
\end{itemize}
\end{frame}

\begin{frame}[fragile]
\frametitle{BVH: SAH}
\usemintedstyle{solarized-light}
\begin{itemize}
\item Для вычисления стоимости всех \begin{math}N-1\end{math} разбиений нам нужно знать площади поверхности соответствующих AABB
\pause
\item AABB всех объектов с первого до K-ого можно вычислить за один проход для всех K, как префиксную сумму:
\pause
\begin{itemize}
\item \mintinline{cpp}|aabb_left[0] = empty|
\pause
\item \mintinline{cpp}|aabb_left[i] = union(aabb_left[i-1], object[i-1].aabb)|
\end{itemize}
\pause
\item Аналогично можно вычислить AABB всех объектов от K-ого до последнего за один проход в обратном порядке
\end{itemize}
\end{frame}

\begin{frame}[fragile]
\frametitle{BVH: binned SAH}
\usemintedstyle{solarized-light}
\begin{itemize}
\item Итоговая стоимость разбиения вершины \begin{math}O(n\cdot \log n)\end{math} (сортировка) и построения дерева \begin{math}O(n\cdot \log^2 n)\end{math}
\pause
\item Построение BVH работает в целом достаточно быстро (порядка 100ms на 100k вершин), но хочется быстрее
\pause
\item Binning-эвристика: вместо проверки всех возможных разбиений, проверим фиксированное количество (напр. 8 или 16) разбиений с фиксированным шагом
\pause
\item Большая серия статей про построение BVH для рейтрейсинга: \href{https://jacco.ompf2.com/2022/04/13/how-to-build-a-bvh-part-1-basics/}{\texttt{jacco.ompf2.com/2022/04/13/how-to-build-a-bvh-part-1-basics}}
\end{itemize}
\end{frame}

\begin{frame}[fragile]
\frametitle{BVH: применение}
\begin{itemize}
\item При вычислении пересечения луча со сценой, проходим отдельно по всем бесконечным объектам, а для конечных объектов обходим BVH
\pause
\item При multiple importance sampling'е источников света, для вычисления правильной вероятности направления, нам нужно найти все источники света, пересекающие луч из текущей точки в этом направлении
\pause
\item Лучше для источников света завести отдельную, независимую BVH (и \textbf{не делать} early-out при её обходе, так как нам нужны \textbf{все} пересечения, а не ближайшее)
\end{itemize}
\end{frame}

\begin{frame}[fragile]
\frametitle{BVH: BLAS и TLAS}
\begin{itemize}
\item Часто неудобно иметь одну BVH на все треугольники сцены, если какие-то объекты двигаются или анимируются
\pause
\item В таком случае разделяют
\pause
\begin{itemize}
\item \textit{Bottom-level accelerating structure (BLAS)} -- BVH, которая строится для конкретного объекта в его локальной системе координат
\pause
\item \textit{Top-level accelerating structure (TLAS)} -- BVH, которая строится не для треугольников, а для самих объектов, в системе координат сцены
\end{itemize}
\pause
\item Тогда нужен двухуровневый проход по BVH
\pause
\begin{itemize}
\item Вычисляем, какие объекты могут пересекаться лучом, используя TLAS
\pause
\item Вычисляем, какие треугольники пересекаются лучом, используя BLAS каждого объекта
\end{itemize}
\end{itemize}
\end{frame}

\begin{frame}[fragile]
\frametitle{Распараллеливание}
\begin{itemize}
\item Чтобы распараллелить рендеринг на несколько потоков, нужно сначала разбить его на независимые подзадачи
\pause
\item Можно выбрать разную гранулярность подзадач:
\pause
\begin{itemize}
\item Подзадача -- большой кусок изображения
\pause
\item Подзадача -- одна строка изображения
\pause
\item Подзадача -- маленький (8x8) тайл изображения
\pause
\item Подзадача -- один пиксель
\pause
\item Подзадача -- один семпл в пикселе
\pause
\item Подзадача -- один луч
\end{itemize}
\end{itemize}
\end{frame}

\begin{frame}[fragile]
\frametitle{Распараллеливание: гранулярность}
\begin{itemize}
\item Подбирать гранулярность нужно из нескольких соображений
\pause
\item Нужно максимизировать равномерность нагрузки на разные подзадачи (чтобы не простаивали потоки, справившиеся раньше других)
\pause
\begin{itemize}
\item \begin{math}\Longrightarrow\end{math} Разбиение на большие куски или даже строки изображения нам не подходит: в одном куске может оказаться намного больше объектов, чем в другом
\end{itemize}
\pause
\item Нужно минимизировать необходимость синхронизации
\pause
\begin{itemize}
\item \begin{math}\Longrightarrow\end{math} Разбиение на отдельные лучи не подходит: лучу придётся ждать рекурсивно вызванный луч
\end{itemize}
\pause
\item Нужно максимизировать локальность данных
\pause
\begin{itemize}
\item \begin{math}\Longrightarrow\end{math} Разбиение на отдельные семплы не подходит: все семплы одного пикселя будут обращаться +/- к одним и тем же данным, лучше их сгруппировать в одну подзадачу
\end{itemize}
\end{itemize}
\end{frame}

\begin{frame}[fragile]
\frametitle{Распараллеливание: тайлы}
\begin{itemize}
\item Стандартный выбор при рейтрейсинге -- гранулярность на уровне пикселей или небольших (8x8) тайлов пикселей
\pause
\item Пиксели тайла объединяются в одну подзадачу из тех же соображений, что и семплы одного пикселя
\pause
\item Выигрыш от тайлов становится заметен на очень больших сценах (десятки-сотни миллионов вершин), когда сцена уже не влезает ни в какие кеши CPU
\end{itemize}
\end{frame}

\begin{frame}[fragile]
\frametitle{Распараллеливание: реализация}
\usemintedstyle{solarized-light}
\begin{itemize}
\item Нужно придумать уникальное описание конкретной подзадачи (напр. ID пикселя), и завести синхронизированную очередь всех подзадач
\pause
\item Нужно создать N потоков (где N -- количество ядер процессора, напр. \mintinline{cpp}|std::thread::hardware_concurrency|), которые будут по одной доставать подзадачи из очереди и выполнять их
\pause
\item В C++ это можно легко сделать с помощью OpenMP (нужно будет добавить его в CMakeLists) директивой \mintinline{cpp}|#pragma omp parallel for|
\end{itemize}
\end{frame}

\begin{frame}[fragile]
\frametitle{Распараллеливание: pitfalls}
\usemintedstyle{solarized-light}
\begin{itemize}
\item Нужно убедиться, что все подзадачи независимы, и не перезатирают данные друг друга (напр. разные потоки не пишут в один и тот же пиксель)
\pause
\item Нужно убедиться в потокобезопасности сцены и BVH (их непросто сделать потоконебезопасными, -- например, неаккуратно что-нибудь кешируя)
\pause
\item Нужно убедиться, что разные потоки используют разные и по-разному проинициализированные генераторы случайных чисел (можно их проинициализировать, используя ID потока или ID подзадачи)
\end{itemize}
\end{frame}

\begin{frame}[fragile]
\frametitle{Оптимизации: результат}
\slideimage{dragon_100k.png}
\centerline{100k faces, 1024x1024, 512 spp, 180 s}
\end{frame}

\begin{frame}[fragile]
\frametitle{Оптимизации: результат}
\slideimage{dragon_100k_metal.png}
\centerline{100k faces, 1024x1024, 512 spp, 160 s}
\end{frame}

\begin{frame}[fragile]
\frametitle{Оптимизации: результат}
\slideimage{dragon_100k_glass.png}
\centerline{100k faces, 1024x1024, 512 spp, 300 s}
\end{frame}

\begin{frame}[fragile]
\frametitle{Оптимизации: результат}
\slideimage{dragon_100k_glow.png}
\centerline{100k faces, 1024x1024, 512 spp, 1000 s}
\end{frame}

\begin{frame}[fragile]
\frametitle{Оптимизации: результат}
\slideimage{spheres.png}
\end{frame}

\begin{frame}[fragile]
\frametitle{Оптимизации: результат}
\slideimage{cubes.png}
\end{frame}

\end{document}