% (c) Nikita Lisitsa, lisyarus@gmail.com, 2024

\documentclass[10pt,handout]{beamer}

\usepackage[T2A]{fontenc}
\usepackage[russian]{babel}
\usepackage{minted}

\usepackage{graphicx}
\graphicspath{ {./images/} }

\usepackage{adjustbox}

\usepackage{color}
\usepackage{soul}

\usepackage{hyperref}

\usetheme{metropolis}

\definecolor{red}{rgb}{1,0,0}
\definecolor{green}{rgb}{0,0.5,0}
\definecolor{blue}{rgb}{0,0,1}
\definecolor{codebg}{RGB}{29,35,49}
\definecolor{lightbg}{RGB}{253,246,227}
\setminted{fontsize=\footnotesize}

\makeatletter
\newcommand{\slideimage}[1]{
  \begin{figure}
    \begin{adjustbox}{width=\textwidth, totalheight=\textheight-2\baselineskip-2\baselineskip,keepaspectratio}
      \includegraphics{#1}
    \end{adjustbox}
  \end{figure}
}
\makeatother

\title{Фотореалистичный рендеринг\quad\quad\quad\quad\quad\quad \textit{(aka raytracing)}}
\subtitle{Практика 2}
\date{2024}

\setbeamertemplate{footline}[frame number]

\begin{document}

\frame{\titlepage}

\begin{frame}[fragile]
\frametitle{Описание практики}
В этой практике нужно добавить:
\begin{itemize}
\item Источники света
\item Тени
\item Три вида материалов
\item Коррекцию цвета
\end{itemize}
\end{frame}

\begin{frame}[fragile]
\frametitle{Описание материалов}
\begin{itemize}
\item В данной практике каждый материал -- или диффузный, или металл, или прозрачный диэлектрик
\pause
\item Для диффузного материала нужно вычислить вклад всех видов света (ambient + все источники) и домножить на цвет объекта
\pause
\item Для металла нужно вычислить цвет отражённого луча и домножить на цвет объекта
\pause
\item Для диэлектрика нужно вычислить цвет отражённого и преломлённого лучей, и смешать их по закону Френеля
\pause
\begin{itemize}
\item Не забудьте про случай полного внутреннего рассеяния
\pause
\item Не забудьте, что для диэлектриков домножать на цвет объекта нужно только преломлённый свет, и только если мы смотрим снаружи объекта
\end{itemize}
\end{itemize}
\end{frame}

\begin{frame}[fragile]
\frametitle{Коррекция цвета}
\begin{itemize}
\item Используем ACES tone-mapping по формуле из лекции, и затем гамма-коррекцию
\pause
\item \alert{\textbf{N.B.:}} Всё это \textbf{до} перевода из диапазона [0, 1] в пиксели 0..255
\end{itemize}
\end{frame}

\begin{frame}
\frametitle{Формат сцены: новые команды}
\begin{itemize}
\item \texttt{RAY\_DEPTH <depth>} -- максимальная глубина рекурсии
\item \texttt{AMBIENT\_LIGHT <red> <green> <blue>} -- интенсивность глобального ambient освещения
\item \texttt{NEW\_LIGHT} -- добавить новый источник света (последующие команды будут задавать его свойства)
\item \texttt{LIGHT\_INTENSITY <red> <green> <blue>} -- интенсивность источника света
\item \texttt{LIGHT\_DIRECTION <x> <y> <z>} -- нормированный вектор направления на источник света (только для направленных источников)
\item \texttt{LIGHT\_POSITION <x> <y> <z>} -- позиция источника света (только для точечных источников)
\item \texttt{LIGHT\_ATTENUATION <c0> <c1> <c2>} -- коэффициенты затухания источника света (только для точечных источников)
\end{itemize}
\end{frame}

\begin{frame}
\frametitle{Формат сцены: новые команды}
\begin{itemize}
\item \texttt{METALLIC} -- материал последнего добавленного объекта -- металл
\item \texttt{DIELECTRIC} -- материал последнего добавленного объекта -- диэлектрик
\item Если не указаны \texttt{METALLIC} и \texttt{DIELECTRIC}, материал считается диффузным
\item \texttt{IOR} -- коэффициент преломления для последнего добавленного объекта (имеет смысл только для диэлектриков)
\end{itemize}
\end{frame}


\begin{frame}[fragile]
\frametitle{Пример сцены}
\href{https://github.com/lisyarus/raytracing-course-slides/tree/trunk/example_scenes/practice2.txt}{\texttt{github.com/lisyarus/raytracing-course-slides/tree/trunk/example\_scenes/practice2.txt}}
\end{frame}

\begin{frame}
\frametitle{Пример сцены}
\begin{figure}
\slideimage{result.png}
\end{figure}
\end{frame}

\end{document}
