% (c) Nikita Lisitsa, lisyarus@gmail.com, 2024

\documentclass[10pt,handout]{beamer}

\usepackage[T2A]{fontenc}
\usepackage[russian]{babel}
\usepackage{minted}

\usepackage{graphicx}
\graphicspath{ {./images/} }

\usepackage{adjustbox}

\usepackage{color}
\usepackage{soul}

\usepackage{hyperref}

\usetheme{metropolis}

\definecolor{red}{rgb}{1,0,0}
\definecolor{green}{rgb}{0,0.5,0}
\definecolor{blue}{rgb}{0,0,1}
\definecolor{codebg}{RGB}{29,35,49}
\definecolor{lightbg}{RGB}{253,246,227}
\setminted{fontsize=\footnotesize}

\makeatletter
\newcommand{\slideimage}[1]{
  \begin{figure}
    \begin{adjustbox}{width=\textwidth, totalheight=\textheight-2\baselineskip-2\baselineskip,keepaspectratio}
      \includegraphics{#1}
    \end{adjustbox}
  \end{figure}
}
\makeatother

\title{Фотореалистичный рендеринг\quad\quad\quad\quad\quad\quad \textit{(aka raytracing)}}
\subtitle{Практика 5}
\date{2024}

\setbeamertemplate{footline}[frame number]

\begin{document}

\frame{\titlepage}

\begin{frame}[fragile]
\frametitle{Описание практики}
Задачи этой практики:
\begin{itemize}
\item Поддержать треугольники в качестве геометрических примитивов
\item Реализовать BVH с использованием SAH (можно binning, можно полный перебор)
\item Использовать BVH для пересечения луча со сценой и для вычисления pdf при light sampling'е
\end{itemize}
\end{frame}

\begin{frame}
\frametitle{Формат сцены: новые команды}
\begin{itemize}
\item \texttt{TRIANGLE X1 Y1 Z1 X2 Y2 Z2 X3 Y3 Z3} -- (после команды \texttt{NEW\_PRIMITIVE}) задать геометрию объекта -- треугольник с заданными координатами вершин
\end{itemize}
\end{frame}

\begin{frame}[fragile]
\frametitle{Примеры сцены}
\begin{scriptsize}
\href{https://github.com/lisyarus/raytracing-course-slides/tree/trunk/example_scenes/practice5_1.txt}{\texttt{slides/tree/trunk/example\_scenes/practice3\_1.txt}}
\href{https://github.com/lisyarus/raytracing-course-slides/tree/trunk/example_scenes/practice5_2.txt}{\texttt{slides/tree/trunk/example\_scenes/practice3\_2.txt}}
\href{https://github.com/lisyarus/raytracing-course-slides/tree/trunk/example_scenes/practice5_dragon_10k.txt}{\texttt{slides/tree/trunk/example\_scenes/practice5\_dragon\_10k.txt}}
\href{https://github.com/lisyarus/raytracing-course-slides/tree/trunk/example_scenes/practice5_dragon_100k.txt}{\texttt{slides/tree/trunk/example\_scenes/practice5\_dragon\_100k.txt}}
\href{https://github.com/lisyarus/raytracing-course-slides/tree/trunk/example_scenes/practice5_dragon_100k_metal.txt}{\texttt{slides/tree/trunk/example\_scenes/practice5\_dragon\_100k\_metal.txt}}
\href{https://github.com/lisyarus/raytracing-course-slides/tree/trunk/example_scenes/practice5_dragon_100k_glass.txt}{\texttt{slides/tree/trunk/example\_scenes/practice5\_dragon\_100k\_glass.txt}}
\href{https://github.com/lisyarus/raytracing-course-slides/tree/trunk/example_scenes/practice5_dragon_100k_glow.txt}{\texttt{slides/tree/trunk/example\_scenes/practice5\_dragon\_100k\_glow.txt}}
\end{scriptsize}
\end{frame}

\begin{frame}
\frametitle{Пример сцены №1}
\begin{figure}
\slideimage{practice5_1.png}
\end{figure}
\centerline{6 s}
\end{frame}

\begin{frame}
\frametitle{Пример сцены №2}
\begin{figure}
\slideimage{practice5_2.png}
\end{figure}
\centerline{14 s}
\end{frame}

\begin{frame}
\frametitle{Пример сцены dragon 10k}
\begin{figure}
\slideimage{dragon_10k.png}
\end{figure}
\centerline{120 s}
\end{frame}

\begin{frame}
\frametitle{Пример сцены dragon 100k}
\begin{figure}
\slideimage{dragon_100k.png}
\end{figure}
\centerline{180 s}
\end{frame}

\begin{frame}
\frametitle{Пример сцены dragon 100k metal}
\begin{figure}
\slideimage{dragon_100k_metal.png}
\end{figure}
\centerline{160 s}
\end{frame}

\begin{frame}
\frametitle{Пример сцены dragon 100k glass}
\begin{figure}
\slideimage{dragon_100k_glass.png}
\end{figure}
\centerline{300 s}
\end{frame}

\begin{frame}
\frametitle{Пример сцены dragon 100k glow}
\begin{figure}
\slideimage{dragon_100k_glow.png}
\end{figure}
\centerline{1000 s}
\end{frame}

\end{document}
