% (c) Nikita Lisitsa, lisyarus@gmail.com, 2024

\documentclass[10pt,handout]{beamer}

\usepackage[T2A]{fontenc}
\usepackage[russian]{babel}
\usepackage{minted}

\usepackage{graphicx}
\graphicspath{ {./images/} }

\usepackage{adjustbox}

\usepackage{color}
\usepackage{soul}

\usepackage{hyperref}

\usetheme{metropolis}

\definecolor{red}{rgb}{1,0,0}
\definecolor{green}{rgb}{0,0.5,0}
\definecolor{blue}{rgb}{0,0,1}
\definecolor{codebg}{RGB}{29,35,49}
\definecolor{lightbg}{RGB}{253,246,227}
\setminted{fontsize=\footnotesize}

\makeatletter
\newcommand{\slideimage}[1]{
  \begin{figure}
    \begin{adjustbox}{width=\textwidth, totalheight=\textheight-2\baselineskip-2\baselineskip,keepaspectratio}
      \includegraphics{#1}
    \end{adjustbox}
  \end{figure}
}
\makeatother

\title{Фотореалистичный рендеринг\quad\quad\quad\quad\quad\quad \textit{(aka raytracing)}}
\subtitle{Практика 1}
\date{2024}

\setbeamertemplate{footline}[frame number]

\begin{document}

\frame{\titlepage}

\begin{frame}[fragile]
\frametitle{Напоминание об автотестах}
\begin{itemize}
\item Каждая практика -- программа, по описанию входной сцены генерирующая картинку (\textit{`рендер'})
\pause
\item Автотестированием занимается телеграм-бот \texttt{mkn-raytracing-2024-bot}
\pause
\item Боту нужно послать команду вида
\usemintedstyle{solarized-light}
\begin{minted}[bgcolor=lightbg]{bash}
/submit practice1 https://github.com/vanya/practice1.git
  [ path-inside-repo [ branch ] ]
\end{minted}
\pause
\item В репозитории по указанному пути должны быть скрипты \texttt{build.sh} и \texttt{run.sh}
\pause
\item Скрипт \texttt{run.sh} запускается с двумя параметрами: путь до файла с описанием сцены, и путь до выходного файла с картинкой
\end{itemize}
\end{frame}

\begin{frame}
\frametitle{Формат сцены}
\begin{itemize}
\item Текстовый формат, простой для ручного парсинга
\pause
\item Каждая строка файла -- отдельная команда, задающая какие-то параметры сцены
\pause
\item Команды выглядят как \texttt{COMMAND\_NAME <arg1> <arg2> ...}, все аргументы -- вещественные числа
\pause
\item Неизвестные команды \textit{нужно пропускать}
\end{itemize}
\end{frame}

\begin{frame}
\frametitle{Формат сцены: описание команд}
\begin{itemize}
\item \texttt{DIMENSIONS <width> <height>} -- размеры изображения, которое нужно сгенерировать
\item \texttt{BG\_COLOR <red> <green> <blue>} -- цвет фона сцены (значения в диапазоне [0..1])
\item \texttt{CAMERA\_POSITION <x> <y> <z>} -- координаты позиции камеры
\item \texttt{CAMERA\_RIGHT <x> <y> <z>} -- ось `вправо' камеры
\item \texttt{CAMERA\_UP <x> <y> <z>} -- ось `вверх' камеры
\item \texttt{CAMERA\_FORWARD <x> <y> <z>} -- ось `вперёд' камеры
\item \texttt{CAMERA\_FOV\_X <angle>} -- угол обзора камеры по ширине (в радианах)
\item \alert{\textbf{N.B.:}} \begin{math}fov_Y\end{math} вычисляется на основе \begin{math}fov_X, width, height\end{math} (см. слайды лекции)
\end{itemize}
\end{frame}

\begin{frame}
\frametitle{Формат сцены: описание команд}
\begin{itemize}
\item \texttt{NEW\_PRIMITIVE} -- добавить новый объект в список объектов сцены (дальнейшие команды описывают последний созданный объект)
\item \texttt{PLANE <nx> <ny> <ny>} -- задать геометрию объекта -- плоскость с заданным вектором нормали
\item \texttt{ELLIPSOID <rx> <ry> <ry>} -- задать геометрию объекта -- эллипсоид с заданными радиусами
\item \texttt{BOX <sx> <sy> <sy>} -- задать геометрию объекта -- параллелепипед с заданными размерами (как в лекции)
\item \texttt{POSITION <x> <y> <y>} -- координаты центра объекта, по умолчанию -- \begin{math}(0, 0, 0)\end{math}
\item \texttt{ROTATION <x> <y> <y> <w>} -- кватернион вращения объекта, по умолчанию -- \begin{math}(0, 0, 0, 1)\end{math}
\item \texttt{COLOR <red> <green> <blue>} -- цвет объекта
\end{itemize}
\end{frame}

\begin{frame}[fragile]
\frametitle{Пример сцены}
\usemintedstyle{solarized-light}
\begin{minted}[bgcolor=lightbg,fontsize=\scriptsize]{text}
DIMENSIONS 640 480

BG_COLOR 0 0 0.5

CAMERA_POSITION 0 1.5 0
CAMERA_RIGHT 1 0 0
CAMERA_UP 0 1 0
CAMERA_FORWARD 0 0 -1
CAMERA_FOV_X 1.5708

NEW_PRIMITIVE
ELLIPSOID 2 2 2
POSITION -1 1 -5
COLOR 1 0 0

NEW_PRIMITIVE
PLANE 0 1 0
COLOR 0 1 0

NEW_PRIMITIVE
BOX 0.5 0.5 0.5
POSITION 1.5 2.5 -3
ROTATION 0.31246 0.15623 0.15623 0.92388
COLOR 1 1 0
\end{minted}
\end{frame}

\begin{frame}
\frametitle{Пример сцены}
\begin{figure}
\slideimage{result.png}
\end{figure}
\end{frame}

\begin{frame}
\frametitle{Формат изображения}
\begin{itemize}
\item NetPBM P6: текстово-бинарный формат, удобный для чтения и записи
\pause
\item Имеет расширение файла \texttt{.ppm} (вам его дописывать не нужно)
\pause
\item Первая строка файла: \texttt{P6}
\item Вторая строка файла: \texttt{<width> <height>} (нужно подставить ширину и высоту изображения)
\item Третья строка файла: \texttt{255} (максимальное значение цветовых компонент, у нас всегда 255)
\item После этого -- пиксели изображения в сыром, бинарном формате (всего \texttt{width * height * 3} байт)
\end{itemize}
\end{frame}

\begin{frame}[fragile]
\frametitle{Советы}
\begin{itemize}
\item Начните с парсинга, и реализуйте команды по одной
\pause
\item Удобно начать тестировать с камеры в центре координат и сферы, на которую смотрит камера (должен появиться кружок на экране)
\pause
\item Удобно разбить код на функции, например
\usemintedstyle{solarized-light}
\begin{minted}[bgcolor=lightbg]{cpp}
ray generate_ray(camera, pixel_coord);
optional<float> intersection(ray, plane);
optional<float> intersection(ray, sphere);
optional<float> intersection(ray, box);
optional<float> intersection(ray, object);
optional<pair<float, color>> intersection(ray, scene);
color raytrace(scene, ray);
image generate_image(scene);
\end{minted}
\end{itemize}
\end{frame}

\end{document}
